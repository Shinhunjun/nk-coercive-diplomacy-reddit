\section{Discussion}
\textbf{Asymmetric Persistence in Diplomatic Framing: } Our findings reveal a clear asymmetry in how diplomatic events shape online discourse. The 2018 Singapore Summit produced substantial shifts toward diplomacy-oriented framing across multiple levels of analysis, including individual posts (+0.85 to +1.28 in DiD estimates), relational structures (THREAT edges: 48.4\% $\rightarrow$ 28.0\%), and community organization (THREAT communities: 52.0\% $\rightarrow$ 32.7\%). In contrast, the subsequent failure of the 2019 Hanoi Summit triggered only partial reversals. Rather than returning to pre-summit baselines, discourse stabilized at an intermediate state between threat-dominant and diplomacy-oriented narratives.
We refer to this pattern as \textbf{asymmetric persistence}, borrowing the term ``ratchet effect'' as a descriptive metaphor rather than a claim of strict irreversibility. Across content-level framing (reversal ratio = 0.43), edge-level relationships (0.11), and community-level orientations (0.24), the magnitude of reversal following diplomatic failure was consistently smaller than the magnitude of the initial shift following diplomatic success. Notably, this pattern does not hold for sentiment, which fully reverted after the Hanoi Summit. This divergence suggests that diplomatic events reshape \textit{how} adversaries are discussed more durably than \textit{how they are emotionally evaluated}. Importantly, while our causal identification is defined at the post level, comment-level findings are interpreted as evidence of propagation and external validity rather than as independent causal estimates. This suggests that diplomatic events can restructure not only agenda-setting content but also the interpretive dynamics of online publics.

One plausible explanation for this asymmetry lies in the structural reorganization of discourse. Our network analysis indicates that diplomatic engagement consolidates discussion around a smaller set of central actors and themes, increasing narrative connectivity and reducing fragmentation. Once such connections are established (linking adversaries to negotiation, dialogue, and institutional processes), they appear to persist even when subsequent events undermine diplomatic optimism. In this sense, diplomatic success may introduce new interpretive linkages that are not immediately undone by later failure.
Importantly, we do not claim that framing shifts are permanent or irreversible. Rather, our results suggest that framing responds to positive diplomatic engagement in a more \textit{structurally persistent} manner than sentiment. This asymmetry highlights the value of distinguishing between affective reactions and interpretive structures when evaluating the public impact of diplomacy. While sentiment captures short-lived emotional responses to events, framing reflects deeper narrative orientations that can endure beyond immediate political outcomes~\cite{bestvater2023sentiment}. This divergence is further reinforced by comment-level patterns: even when affective responses among audiences remain elastic, the relational and thematic structure of discourse continues to privilege diplomatic interpretations over threat-oriented ones.
We acknowledge that alternative mechanisms could potentially explain the observed asymmetric persistence. \textit{Agenda saturation}, for example, might suggest that diplomatic themes simply became part of the general news cycle, making reversal less likely due to diminished novelty. \textit{Topic fatigue} could imply that users lost interest in adversarial framing altogether, rather than genuinely shifting their interpretive schemas. A third alternative, \textit{background knowledge absorption}, posits that diplomatic frames became embedded as default contextual knowledge rather than actively maintained discourse positions. However, several features of our findings favor a structural ratchet interpretation over these alternatives. First, if agenda saturation or topic fatigue were driving, we would expect \textit{uniform} decline across all framing categories---yet THREAT framing specifically decreased while DIPLOMACY framing increased. Second, the network-level reorganization (increased centralization, reduced fragmentation) indicates active structural change rather than passive attention decay. Third, the differential behavior of sentiment versus framing is inconsistent with general fatigue explanations: sentiment fully reverted while framing persisted, suggesting that interpretive frames---rather than attention levels---are the locus of durability.

\textbf{Multi-Level Discourse Reorganization:} Beyond content-level shifts, our graph-based network analysis reveals that diplomatic events reorganize discourse at multiple levels~\cite{vandenhole2025discourse}. The Singapore Summit produced coordinated shifts: threat edges dropped 20pp while diplomacy communities nearly doubled (24.6\% $\rightarrow$ 42.5\%). Following Hanoi, humanitarian and economic communities expanded, suggesting agenda broadening rather than simple reversion. The predominance of neutral discourse in comments highlights that the observed shifts are not driven by wholesale opinion change, but by selective reorganization of how salient frames are connected within public discussion.
% \mohit{No need for the following subsection, Blinding it} \jun{Thanks}
% \subsection{Robustness and Methodological Consistency}
% Our findings are strengthened by methodological consistency. The same GPT-4o-mini framing model was applied across all analyses (post classification, edge classification, and community classification) ensuring that observed patterns reflect genuine discourse dynamics rather than measurement artifacts.
% The multi-control Difference-in-Differences design further supports causal interpretation, with Iran and China satisfying parallel trends assumptions while Russia exhibits partial violations that inform boundary conditions.

\textbf{Implications for Policy Communication: } While our findings are based on engaged online communities rather than representative samples, they suggest potential implications for diplomatic communication strategies: high-profile diplomatic initiatives may rapidly alter interpretive frames, even in adversarial contexts. Positive framing exhibits persistence following successful engagement (ratchet effect), and framing effects appear more pronounced than sentiment effects. Failed diplomacy does not fully undo earlier gains, indicating some resilience in discourse shifts. However, these patterns require validation across broader populations before informing policy directly.
% \begin{enumerate}
%     \item 
%     \item 
%     \item Framing effects are more pronounced than sentiment effects, suggesting policymakers should prioritize narrative positioning
%     \item Failed diplomacy does not fully undo earlier gains, indicating some resilience in discourse shifts
% \end{enumerate}

\textbf{Limitations \& Future Work.}
Several limitations warrant consideration. First, our analysis is limited to English-language Reddit posts, which may not capture discourse dynamics on non-English platforms or in regions with minimal Reddit usage. Furthermore, Reddit users, who skew younger and more politically engaged, are not representative of the general public.
Second, while we employed a robust multi-control design, Russia's exclusion from the P1$\rightarrow$P2 framing analysis due to parallel trends violation ($p=0.01$) reduces our control group to two countries for the Singapore period. This limitation affects the robustness of our counterfactual estimates, as confounding control relies on having multiple comparison groups. We recommend applying sensitivity analysis frameworks such as HonestDiD~\cite{rambachan2023honest} to quantify robustness under varying parallel trends assumptions.
Third, our aggregate analysis does not distinguish whether rhetorical shifts stem from individual belief changes or platform compositional churn (i.e., new users entering the conversation). Future work should employ user-level longitudinal models to disentangle these mechanisms.
Fourth, our 10-month post-Hanoi window captures \textit{short-term} asymmetric persistence; examining long-term durability against subsequent shocks (e.g., COVID-19) remains an open question.
% \mohit{Change this now since you did comment analysis, but say we only did top level and the whole structure of the comments can give more inisghts or something like that}Finally, our causal analysis focuses on original posts; while we include a supplementary comment-level extension, we do not model comment threads as independent causal units. Future research should extend this framework to comment threads to capture deeper deliberative dynamics.

\textbf{Ethical Considerations \& Broader Impact.} This study uses publicly available Reddit data accessed through the Arctic Shift API~\cite{arcticshift2024}. No personally identifiable information was collected, and all posts analyzed were publicly shared by users. While analyzing discourse dynamics supports transparent policymaking, we acknowledge potential risks: automated framing tools could be repurposed for computational propaganda. We advocate for their responsible use to detect rather than generate manipulative narratives.
% The research was conducted in accordance with Reddit's terms of service and does not require IRB approval under U.S. federal guidelines for publicly available data.
% \textbf{.}  
% While such tools can inform evidence-based diplomacy and public communication strategies, they also carry risks: automated framing analysis could be misused for propaganda detection or narrative manipulation. We encourage responsible deployment with transparency about methodological limitations.

\textbf{Reproducibility.} Code and anonymized data are available at: \url{https://anonymous.4open.science/r/Asymmetric-Persistence-in-Online-Framing-of-North-Korea-F0F7}. 
% To comply with platform policies, we provide post IDs for text retrieval rather than the full raw text.
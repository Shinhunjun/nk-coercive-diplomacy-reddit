\section{Discussion}
\textbf{Asymmetric Persistence in Diplomatic Framing: } Our findings reveal a clear asymmetry in how diplomatic events shape online discourse. The 2018 Singapore Summit produced substantial shifts toward diplomacy-oriented framing across multiple levels of analysis, including individual posts (+0.85 to +1.28 in DiD estimates), relational structures (THREAT edges: 48.4\% $\rightarrow$ 28.0\%), and community organization (THREAT communities: 52.0\% $\rightarrow$ 32.7\%). In contrast, the subsequent failure of the 2019 Hanoi Summit triggered only partial reversals. Rather than returning to pre-summit baselines, discourse stabilized at an intermediate state between threat-dominant and diplomacy-oriented narratives.
We refer to this pattern as \textbf{asymmetric persistence}, borrowing the term ``ratchet effect'' as a descriptive metaphor rather than a claim of strict irreversibility. Across content-level framing (reversal ratio = 0.43), edge-level relationships (0.11), and community-level orientations (0.24), the magnitude of reversal following diplomatic failure was consistently smaller than the magnitude of the initial shift following diplomatic success. Notably, this pattern does not hold for sentiment, which fully reverted after the Hanoi Summit. This divergence suggests that diplomatic events reshape \textit{how} adversaries are discussed more durably than \textit{how they are emotionally evaluated}.
One plausible explanation for this asymmetry lies in the structural reorganization of discourse. Our network analysis indicates that diplomatic engagement consolidates discussion around a smaller set of central actors and themes, increasing narrative connectivity and reducing fragmentation. Once such connections are established (linking adversaries to negotiation, dialogue, and institutional processes), they appear to persist even when subsequent events undermine diplomatic optimism. In this sense, diplomatic success may introduce new interpretive linkages that are not immediately undone by later failure.
Importantly, we do not claim that framing shifts are permanent or irreversible. Rather, our results suggest that framing responds to positive diplomatic engagement in a more \textit{structurally persistent} manner than sentiment. This asymmetry highlights the value of distinguishing between affective reactions and interpretive structures when evaluating the public impact of diplomacy. While sentiment captures short-lived emotional responses to events, framing reflects deeper narrative orientations that can endure beyond immediate political outcomes~\cite{bestvater2023sentiment}.
We acknowledge that alternative mechanisms could potentially explain the observed asymmetric persistence. \textit{Agenda saturation}, for example, might suggest that diplomatic themes simply became part of the general news cycle, making reversal less likely due to diminished novelty. \textit{Topic fatigue} could imply that users lost interest in adversarial framing altogether, rather than genuinely shifting their interpretive schemas. A third alternative, \textit{background knowledge absorption}, posits that diplomatic frames became embedded as default contextual knowledge rather than actively maintained discourse positions. However, several features of our findings favor a structural ratchet interpretation over these alternatives. First, if agenda saturation or topic fatigue were driving, we would expect \textit{uniform} decline across all framing categories---yet THREAT framing specifically decreased while DIPLOMACY framing increased. Second, the network-level reorganization (increased centralization, reduced fragmentation) indicates active structural change rather than passive attention decay. Third, the differential behavior of sentiment versus framing is inconsistent with general fatigue explanations: sentiment fully reverted while framing persisted, suggesting that interpretive frames---rather than attention levels---are the locus of durability.

\textbf{Multi-Level Discourse Reorganization:} Beyond content-level shifts, our graph-based network analysis reveals that diplomatic events reorganize discourse at multiple levels~\cite{vandenhole2025discourse}. The Singapore Summit produced coordinated shifts: threat edges dropped 20pp while diplomacy communities nearly doubled (24.6\% $\rightarrow$ 42.5\%). Following Hanoi, humanitarian and economic communities expanded, suggesting agenda broadening rather than simple reversion.
% \mohit{No need for the following subsection, Blinding it} \jun{Thanks}
% \subsection{Robustness and Methodological Consistency}
% Our findings are strengthened by methodological consistency. The same GPT-4o-mini framing model was applied across all analyses (post classification, edge classification, and community classification) ensuring that observed patterns reflect genuine discourse dynamics rather than measurement artifacts.
% The multi-control Difference-in-Differences design further supports causal interpretation, with Iran and China satisfying parallel trends assumptions while Russia exhibits partial violations that inform boundary conditions.

\textbf{Implications for Policy Communication: } These findings carry implications for diplomatic communication strategies: Firstly, high-profile diplomatic initiatives can rapidly alter interpretive frames, even in adversarial contexts. Secondly, positive framing exhibits persistence following successful engagement (ratchet effect). Thirdly, framing effects are more pronounced than sentiment effects, suggesting that policymakers should prioritize narrative positioning. Finally, failed diplomacy does not fully undo earlier gains, indicating some resilience in discourse shifts.
% \begin{enumerate}
%     \item 
%     \item 
%     \item Framing effects are more pronounced than sentiment effects, suggesting policymakers should prioritize narrative positioning
%     \item Failed diplomacy does not fully undo earlier gains, indicating some resilience in discourse shifts
% \end{enumerate}

\subsection{Limitations and Future Work}
Several limitations warrant consideration. First, our analysis is limited to English-language Reddit posts, which may not capture discourse dynamics on non-English platforms or in countries where Reddit usage is minimal. 
% Public opinion in East Asia (South Korea, Japan, China) may exhibit fundamentally different framing patterns. 
% Second, our human annotation benchmark, while expert-validated, relies on only two domain experts. Although inter-rater reliability was assessed, a larger annotator pool would provide stronger generalizability for the gold-standard labels. 
Second, Reddit users are not representative of the general public. Third, our work only focuses on original posts, and not on the comment threads, which may contain richer deliberative dynamics.
Fourth, Russia's exclusion from the P1$\rightarrow$P2 framing analysis due to parallel trends violation ($p=0.01$) reduces our control group from three to two for the Singapore Summit period. While this exclusion represents a design-based decision following pre-specified statistical criteria rather than post-hoc selection, it nonetheless limits the robustness of our causal estimates for this period. Future work could employ sensitivity analysis frameworks such as HonestDiD~\cite{rambachan2023honest} to quantify robustness to potential trend deviations.
% and direct user interactions that could reveal additional patterns
% The platform skews younger, more male, and more politically engaged, which may amplify or attenuate diplomatic framing effects compared to broader populations. \mohit{This needs citation, especially about younger, male and politically engaged.}
Lastly, our temporal scope (2017--2019) captures only the initial summit period, with the post-Hanoi period spanning approximately 10 months. While we observe asymmetric persistence within this window, the long-term durability of the ratchet effect beyond 2019 (particularly through subsequent developments like the 2020 COVID pandemic and 2022 missile tests) remains untested. Our findings should therefore be interpreted as evidence of \textit{short-term} asymmetric persistence rather than permanent structural change. 
%\hoonbae{how to explicitly define that only summit affects the online discourse we should find more reference to back it up.} \mohit{I agree, however, we did have control groups and this is a limiation}
% Our analysis focuses on original posts and does not include comment threads, which may contain richer deliberative dynamics and direct user interactions that could reveal additional patterns in how framing evolves through discussion.
% \mohit{also mention analysis of comments can shed new light. Since we did not analyse comments, we should mention that as a limitation}
\textbf{Future Work:} Future work should extend this approach to additional platforms (Twitter/X, news comment sections), incorporate longer time horizons to assess ratchet effect durability, and examine the elite-to-public transmission mechanisms through which diplomatic signaling translates into narrative change. Additionally, analyzing Reddit comment threads could shed new light on how framing spreads and transforms through user discourse.
% Fifth, although our multi-control DID design mitigates confounding, unobserved concurrent events (e.g., domestic political shifts, algorithmic changes on Reddit) may partially influence the observed effects.

\textbf{Ethical Considerations \& Broader Impact.} This study uses publicly available Reddit data accessed through the Pushshift archive. No personally identifiable information was collected, and all posts analyzed were publicly shared by users. This work demonstrates how computational methods can systematically measure public discourse shifts following diplomatic events.
% The research was conducted in accordance with Reddit's terms of service and does not require IRB approval under U.S. federal guidelines for publicly available data.
% \textbf{.}  
% While such tools can inform evidence-based diplomacy and public communication strategies, they also carry risks: automated framing analysis could be misused for propaganda detection or narrative manipulation. We encourage responsible deployment with transparency about methodological limitations.

\textbf{Reproducibility.} Code and anonymized data are available at: \url{https://anonymous.4open.science/r/Asymmetric-Persistence-in-Online-Framing-of-North-Korea-F0F7}. 
% To comply with platform policies, we provide post IDs for text retrieval rather than the full raw text.
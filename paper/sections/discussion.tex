%==============================================================================
% DISCUSSION
%==============================================================================
\section{Discussion}

\subsection{Asymmetric Effects: The Ratchet Hypothesis}

Our findings reveal a notable asymmetry in how diplomatic events shape online discourse. The Singapore Summit produced substantial shifts in framing, with threat-oriented content declining and diplomacy-oriented content rising across multiple levels: individual posts (+0.85 to +1.28 in DID estimates), network edges (THREAT: 48.4\% $\rightarrow$ 28.0\%), and community structure (THREAT: 52.0\% $\rightarrow$ 32.7\%).

Critically, the Hanoi Summit failure produced only partial reversals. Rather than returning to pre-summit baselines, discourse stabilized at an intermediate state---what we term the ``ratchet effect.'' This pattern suggests that positive reframing generated by diplomatic success exhibits persistence even after subsequent setbacks, potentially due to the establishment of new interpretive schemas that resist full reversal.

\subsection{Sentiment Elasticity vs. Framing Plasticity}

A central contribution of this study lies in distinguishing between \textit{sentiment} and \textit{framing} as distinct dimensions of public discourse with divergent temporal dynamics. Our results indicate that sentiment behaves **elastically**: it improved during the Singapore Summit but fully reverted to baseline following the Hanoi failure (Ratio $\approx$ 1.15). In contrast, framing exhibited **plasticity**: the shift toward diplomatic narratives persisted partially even after the political catalyst was removed (Ratio $\approx$ 0.12--0.43).

This divergence suggests that sentiment reflects transient emotional reactions to news cycles, whereas framing represents deeper cognitive structures—interpretive schemas—that, once reorganized, are resistant to complete reversal. Once an adversary is conceptualized as a potential "negotiating partner" rather than solely an existential threat, this structural understanding appears to linger in public discourse, buffering against the immediate negativity of diplomatic setbacks.

\subsection{Multi-Level Discourse Reorganization}

Beyond content-level shifts, our GraphRAG analysis reveals that diplomatic events reorganize discourse at the network level. The Singapore Summit was associated with:

\begin{itemize}
    \item \textbf{Edge-level shifts:} Threat framing dropped 20pp while diplomacy framing rose 18pp
    \item \textbf{Community-level shifts:} Diplomacy-oriented communities nearly doubled (24.6\% $\rightarrow$ 42.5\%)
    \item \textbf{Discourse diversification:} Following Hanoi, humanitarian (+9.4pp) and economic (+5.1pp) communities expanded, suggesting agenda broadening
\end{itemize}

This multi-level perspective demonstrates that diplomatic events create cascading effects across discourse organization, from individual expressions to thematic communities.

\subsection{Robustness and Methodological Consistency}

Our findings are strengthened by methodological consistency. The same GPT-4o-mini framing model was applied across all analyses---post classification (RQ1), edge classification (RQ2), and community classification (RQ2)---ensuring that observed patterns reflect genuine discourse dynamics rather than measurement artifacts.

The multi-control Difference-in-Differences design further supports causal interpretation, with Iran and China satisfying parallel trends assumptions while Russia exhibits partial violations that inform boundary conditions.

\subsection{Implications for Policy Communication}

These findings carry implications for diplomatic communication strategies:

\begin{enumerate}
    \item High-profile diplomatic initiatives can rapidly alter interpretive frames, even in adversarial contexts
    \item Positive framing exhibits persistence following successful engagement (ratchet effect)
    \item Framing effects are more pronounced than sentiment effects, suggesting policymakers should prioritize narrative positioning
    \item Failed diplomacy does not fully undo earlier gains, indicating some resilience in discourse shifts
\end{enumerate}

\subsection{Limitations and Future Work}

Several limitations warrant consideration. First, Reddit users are not representative of the broader public. Second, although our LLM classification is validated against expert annotations, systematic biases may remain. Third, unobserved concurrent events may partially confound effects despite our multi-control design. Future work could extend this approach to additional platforms, incorporate longitudinal analysis, and examine mechanisms underlying the ratchet effect.

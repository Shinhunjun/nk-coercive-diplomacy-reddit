% Conclusion
\section{Conclusion}

This study provides causal and structural evidence of how high-stakes diplomatic summits affect public discourse on social media. Using a Difference-in-Differences design with LLM-based classification and expert-validated benchmarks, we find that:

\begin{enumerate}
    \item \textbf{Content-Level Shifts (RQ1):} The 2018 Singapore Summit produced significant positive shifts in framing toward North Korea, with effect sizes substantially larger for framing than sentiment
    \item \textbf{Structural Reorganization (RQ2):} Diplomatic engagement reorganized discourse networks, shifting both edge-level framing (threat $\rightarrow$ diplomacy) and community-level thematic structure
    \item \textbf{The Ratchet Effect:} The 2019 Hanoi Summit failure triggered only partial reversals; discourse did not return to pre-summit baselines, suggesting asymmetry in diplomatic effects
    \item \textbf{Discourse Diversification:} Following diplomatic failure, humanitarian and economic frames expanded, indicating agenda broadening rather than simple reversion
\end{enumerate}

Three methodological contributions emerge from this work. First, applying the same LLM framing model across content (RQ1) and network (RQ2) analyses ensures methodological consistency. Second, our multi-control DID design with Iran and China as counterfactuals supports causal interpretation. Third, the GraphRAG-based network analysis reveals discourse dynamics invisible to content-only approaches.

Future work should extend this analysis to additional platforms, examine longer time horizons to assess the durability of the ratchet effect, and investigate mechanisms through which elite diplomatic signaling translates into public narrative change.

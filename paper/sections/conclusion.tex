\section{Conclusion}
This study provides causal and structural evidence of how high-stakes diplomatic summits affect public discourse on social media. Using a Difference-in-Differences (DiD) design with LLM-based classification and expert-validated benchmarks, we find that: The 2018 Singapore Summit produced significant positive shifts in framing toward North Korea, with effect sizes substantially larger for framing than sentiment. Additionally, diplomatic engagement reorganized discourse networks, shifting both edge-level framing (threat $\rightarrow$ diplomacy) and community-level thematic structure. The 2019 Hanoi Summit failure triggered only partial reversals; discourse did not return to pre-summit baselines, suggesting asymmetry in diplomatic effects. Importantly, these patterns propagated to highly visible audience responses. While comment-level emotional sentiment exhibited the same transience observed at the post level (fully reverting after Hanoi), the structural framing and network configuration of audience discourse maintained the asymmetric persistence and ``hostile engagement'' observed in agenda-setting posts, confirming that top-down diplomatic signals can reshape bottom-up narrative structures even when emotional valence is fleeting.


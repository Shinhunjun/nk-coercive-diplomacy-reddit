\section{Results}
\label{res}

\subsection{Content Changes in Online Discourse (RQ1a)}
We first examine whether high-stakes diplomatic events causally reshape the content of online discourse toward North Korea. 
While sentiment captures general affective valence (positive vs. negative), it lacks the granularity to distinguish between substantive policy narratives. Our primary focus is therefore on framing, which explicitly measures the rhetorical shift between military threat and diplomatic engagement. Together, these measures allow us to assess both the emotional tone and the narrative substance of the discourse following diplomatic events. For instance, a post titled ``North Korea complained to the United Nations on Monday about joint military exercises by the United States and South Korea, describing it as the worst ever situation'' exhibits strongly negative sentiment (score: -0.85) due to the alarmist language, yet is correctly classified as \textit{Diplomacy} (confidence: 0.90). This classification reflects that filing a formal complaint through the United Nations represents diplomatic engagement, regardless of the negative tone. This divergence highlights why relying solely on sentiment would overlook the continued diplomatic engagement underlying the critical rhetoric.
To explicitly quantify the comparative sensitivity of these measures, 
In the transition to P2 (Summit Diplomacy), the framing ratio (Diplomacy/Threat) surged (+204\%) much more sharply than sentiment (+100\%). Following the Hanoi failure, both measures collapsed, reinforcing the view that distinguishing framing from sentiment is critical for capturing asymmetric persistence.

\textbf{Impact on Sentiment: } Figure~\ref{fig:sentiment_did} visualizes the DiD estimates. The Singapore Summit (P1$\rightarrow$P2) produced a statistically significant positive shift in sentiment toward North Korea across all control groups (Table~\ref{tab:did_sentiment_summary}), with causal estimates ranging from +0.10 to +0.21 ($p < 0.001$). In contrast, the failure of the Hanoi Summit (P2$\rightarrow$P3) resulted in a significant negative shift in sentiment, ranging from -0.06 to -0.12 ($p < 0.001$). 

While these sentiment shifts are statistically significant, sentiment alone cannot reliably distinguish between threat-oriented and diplomacy-oriented discourse. Both exhibit predominantly negative valence with substantial overlap in their distributions (THREAT: $M = -0.27$, $SD = 0.27$; DIPLOMACY: $M = -0.08$, $SD = 0.34$). Consequently, an improvement in sentiment may reflect either a genuine shift toward diplomatic framing or merely a softening of tone within the same rhetorical frame. This limitation underscores why framing analysis is essential to capture substantive rhetorical changes.

\begin{figure*}[t]
\centering
    \subfloat[Singapore Summit Effect (P1 \textrightarrow\ P2)]{\includegraphics[width=0.38\textwidth]{figures/fig4b_a_singapore.pdf}\label{fig:sentiment_did_a}}
    % \hfill
    \subfloat[Hanoi Summit Effect (P2 \textrightarrow\ P3)]{\includegraphics[width=0.38\textwidth]{figures/fig4b_b_hanoi.pdf}\label{fig:sentiment_did_b}}
\caption{Sentiment Difference-in-Differences Visualization. Pre-treatment trends (P1) are visually parallel across groups, supporting the validity of the DiD design. The bold black dashed line marks the intervention point (summit date).}
\label{fig:sentiment_did}
\end{figure*}

\begin{table}[t]
\centering
\caption{Sentiment Difference-in-Differences Results }
\begin{tabular}{llcc}
\toprule
\textbf{Event} & \textbf{Control} & \textbf{DiD Est.} & \textbf{95\% CI} \\
\midrule
Singapore & China & +0.21*** & [0.18, 0.24] \\
(P1$\rightarrow$P2) & Iran & +0.10*** & [0.06, 0.13] \\
 & Russia & +0.14*** & [0.11, 0.16] \\
\midrule
Hanoi & China & -0.11*** & [-0.14, -0.08] \\
(P2$\rightarrow$P3) & Iran & -0.06** & [-0.10, -0.02] \\
 & Russia & -0.12*** & [-0.15, -0.10] \\
\bottomrule
\multicolumn{4}{l}{\footnotesize *$p<0.05$, **$p<0.01$, ***$p<0.001$} \\
\end{tabular}
\label{tab:did_sentiment_summary}
\end{table}

\textbf{Impact on Framing:} We next examine changes in framing, where higher values indicate diplomacy-oriented narratives and lower values indicate threat-oriented narratives. 
Figure~\ref{fig:did} visualizes the DiD estimates.

\begin{figure*}[t]
\centering
    \subfloat[Singapore Summit Effect (P1 \textrightarrow\ P2)]{\includegraphics[width=0.38\textwidth]{figures/fig4_a_singapore.pdf}\label{fig:did_a}}
    % \hfill
    \subfloat[Hanoi Summit Effect (P2 \textrightarrow\ P3)]{\includegraphics[width=0.38\textwidth]{figures/fig4_b_hanoi.pdf}\label{fig:did_b}}
\caption{Framing Difference-in-Differences Visualization. Pre-treatment trends are visually parallel for China and Iran, validating their use as control groups. Russia is excluded from (A) due to parallel trends violation (p=0.01) but included in (B) where parallel trends are satisfied. The bold black dashed line marks the intervention point (summit date).}
\label{fig:did}
\end{figure*}

\begin{table}[t]
\centering
\caption{Framing Difference-in-Differences Results}
\begin{tabular}{llcc}
\toprule
\textbf{Event} & \textbf{Control} & \textbf{DiD Est.} & \textbf{95\% CI} \\
\midrule
Singapore & China & +1.17*** & [0.64, 1.69] \\
(P1$\rightarrow$P2) & Iran & +0.69* & [0.17, 1.22] \\
\midrule
Hanoi & China & -0.56* & [-1.09, -0.03] \\
(P2$\rightarrow$P3) & Iran & -0.16 & [-0.72, 0.40] \\
& Russia & -0.54 & [-1.15, 0.06] \\
\bottomrule
\multicolumn{4}{l}{\footnotesize * $p<0.05$, ** $p<0.01$, *** $p<0.001$} \\
\multicolumn{4}{l}{\footnotesize Russia excluded from P1$\rightarrow$P2 due to parallel trends violation.} \\
\end{tabular}
\label{tab:did_framing_summary}
\end{table}

The Singapore Summit led to a pronounced shift toward diplomacy-oriented framing (Table~\ref{tab:did_framing_summary}), with DiD estimates of +0.69 to +1.17 across valid control groups. Following the collapse of the Hanoi Summit, framing shifted back toward threat-oriented narratives, with the China control group showing a statistically significant reversal ($-0.56$, $p = 0.048$). However, the remaining control groups (Iran, Russia) showed no significant reversal---a pattern consistent with the asymmetric persistence (ratchet effect) examined below. Notably, the magnitude of framing shifts substantially exceeds that of sentiment changes, indicating that diplomatic events more strongly reshape how North Korea is framed than how it is emotionally evaluated in online discourse.
To verify that these findings are not artifacts of the diplomacy scale construction (THREAT = $-2$, DIPLOMACY = $+2$), we conducted a robustness check using binary outcome indicators. When analyzing the proportion of THREAT-framed posts directly, DiD estimates show significant decreases following the Singapore Summit across all control groups ($-26.9$ pp with China, $p = 0.004$; $-17.0$ pp with Iran, $p = 0.073$; $-29.1$ pp with Russia, $p = 0.001$). Conversely, DIPLOMACY proportions increased significantly with China ($+21.0$ pp, $p < 0.001$). These binary analyses indicate that the main findings are robust to outcome scaling choices. Notably, diplomatic summits primarily reduced threat framing rather than proportionally increasing diplomacy framing, suggesting that the rhetorical shift was characterized more by de-escalation than by active endorsement of diplomatic engagement. % TODO: Find reference for de-escalation vs endorsement framing distinction

\subsection{Asymmetric Effect Persistence (RQ1b)} 
To test whether diplomatic effects exhibit asymmetric persistence (the ratchet effect hypothesis), we compare the magnitude of framing shifts following the Singapore Summit ($|P1 \rightarrow P2|$) against subsequent reversals following the Hanoi failure ($|P2 \rightarrow P3|$). If the asymmetric persistence hypothesis holds, the reversal effect should be significantly smaller than the original effect.
Table~\ref{tab:ratchet_validation} presents bootstrap validation results across multiple dimensions. For framing dimensions, we use a combined score (DIPLOMACY\% $-$ THREAT\%) to capture the overall shift from threat-dominant to diplomacy-dominant discourse. Ratios significantly below 1.0 indicate incomplete reversal.

\begin{table}[t]
\centering
\small
\setlength{\tabcolsep}{3pt}
\caption{Asymmetric Persistence Validation: Unlike sentiment, framing shifts exhibit asymmetric persistence (Ratio $\ll$ 1.0). Content, Edge, and Community refer to framing metrics at the post, relationship, and cluster levels, respectively.}
\resizebox{\columnwidth}{!}{
\begin{tabular}{lcccc}
\toprule
\textbf{Metric} & \textbf{$\Delta$(P1$\to$P2)} & \textbf{$\Delta$(P2$\to$P3)} & \textbf{Ratio} & \textbf{95\% CI} \\
\midrule
Content & 28.9pp & 11.4pp & 0.39* & [0.15, 0.65] \\
Edge & 38.6pp & 4.3pp & 0.11* & [0.01, 0.24] \\
Community & 37.2pp & 1.4pp & 0.24* & [0.01, 0.72] \\
Sentiment & 0.095 & 0.109 & 1.15 & [0.97, 1.35] \\
\bottomrule
\multicolumn{5}{l}{\footnotesize pp = percentage points; * 95\% CI excludes 1.0 (significant asymmetry)} \\
\end{tabular}
}
\label{tab:ratchet_validation}
\end{table}

The results are consistent with the asymmetric persistence hypothesis for framing dimensions. Content framing shows a reversal ratio of 0.39 (95\% CI: [0.15, 0.65]), indicating that the Hanoi failure reversed only 39\% of the Singapore Summit's framing effect. Edge-level framing exhibits an even stronger ratchet-like pattern (ratio = 0.11), and community framing shows similar persistence (ratio = 0.24). Critically, all framing dimensions have 95\% CIs excluding 1.0. In contrast, sentiment shows a ratio of 1.15 (95\% CI including 1.0), indicating that emotional valence fully reverted following diplomatic failure.
The DiD analysis provides additional statistical evidence for this asymmetry. During the Singapore Summit period (P1$\rightarrow$P2), framing shifts were statistically significant across all valid control groups (China: $p < 0.001$; Iran: $p = 0.014$). In contrast, during the Hanoi reversal period (P2$\rightarrow$P3), only one control group (China) showed a marginally significant effect ($p = 0.048$), while Iran ($p = 0.58$) and Russia ($p = 0.087$) showed no significant reversal. This pattern---where diplomatic engagement produces robust, statistically significant shifts but diplomatic failure produces weaker, less consistent reversals---provides evidence consistent with such a dynamic.

\textbf{Summary: }This divergence between framing and sentiment provides important theoretical insight (RQ1a): while affective responses to diplomacy may be transient, interpretive frames—how North Korea is discussed as a threat versus a negotiation partner—exhibit greater persistence. The asymmetric persistence (RQ1b) appears to be specific to framing rather than sentiment, suggesting that diplomatic engagement primarily reshapes cognitive interpretive schemas rather than emotional evaluations.

\subsection{Structural Reorganization of Discourse Networks (RQ2)}
To assess whether observed framing changes are reflected in the organization of discourse itself, we analyze discourse networks constructed from Reddit discussions across the three diplomatic phases. Using a graph-based indexing approach~\cite{edge2024graphrag}, we extract entities and relationships from 10,659 documents, constructing knowledge graphs that capture how actors, events, and concepts are interconnected in public discourse.

\textbf{Network Topology Changes:} Table~\ref{tab:network_topology} presents network-level metrics across the three periods. Despite substantial reductions in nodes and edges from P1 to P3, network density increased markedly (+146\% from P1 to P2), signifying \textit{structural consolidation}: discussion became highly interconnected around core actors (e.g., ``Trump,'' ``Kim,'' ``summit''). The proportion of nodes in the largest connected component increased from 78\% (P1) to 91\% (P2), indicating that diplomatic engagement integrated previously fragmented discourse threads.

\begin{table}[t]
\centering
\caption{Network Topology Metrics Across Diplomatic Periods}
\begin{tabular}{lcccccc}
\hline
\textbf{Metric} & \textbf{P1} & \textbf{P2} & \textbf{P3} & \textbf{P1$\rightarrow$P2} & \textbf{P2$\rightarrow$P3} \\
\hline
Nodes & 2,656 & 1,043 & 879 & $-60.7\%$ & $-15.7\%$ \\
Edges & 4,552 & 1,726 & 1,429 & $-62.1\%$ & $-17.2\%$ \\
Density ($\times10^{-3}$) & 1.3 & 3.2 & 3.7 & $+146\%$ & $+16\%$ \\
Avg Degree & 3.43 & 3.31 & 3.25 & $-3.4\%$ & $-1.8\%$ \\
Clustering Coef. & 0.215 & 0.198 & 0.210 & $-7.9\%$ & $+6.0\%$ \\
Components & 25 & 23 & 16 & $-8.0\%$ & $-30.4\%$ \\
\hline
\end{tabular}
\label{tab:network_topology}
\end{table}


\textbf{Relationship Framing Shifts: }
Table~\ref{tab:relationship_framing} reveals a pronounced structural inversion. In P1, threat-oriented relationships dominated (48.4\%), but following Singapore (P2), threat framing dropped to 28.0\% ($-20.4$ pp) while diplomacy framing nearly doubled ($+18.3$ pp). 
\begin{table}[t]
\centering
\setlength{\tabcolsep}{4pt}
\caption{Relationship Framing Distribution}
\small
\begin{tabular}{lccccc}
\hline
\textbf{Frame} & \textbf{P1} & \textbf{P2} & \textbf{P3} & \textbf{$\Delta_1$} & \textbf{$\Delta_2$} \\
\hline
Threat & 61.5\% & 42.4\% & 40.9\% & \textbf{$-19$pp} & $-1$pp \\
Peace & 7.8\% & 24.7\% & 18.6\% & \textbf{$+17$pp} & $-6$pp \\
Neutral & 30.8\% & 32.9\% & 40.5\% & $+2$pp & $+8$pp \\
\hline
\end{tabular}
\begin{flushleft}
\footnotesize{$\Delta_1$: P1$\rightarrow$P2; $\Delta_2$: P2$\rightarrow$P3; pp = percentage points}
\end{flushleft}
\label{tab:relationship_framing}
\end{table}

Critically, the Hanoi failure (P3) triggered only partial reversion: threat framing remained low (25.6\%) and diplomacy framing decreased modestly ($-6.7$ pp). This asymmetric pattern suggests that once diplomatic connections are established in the public's conceptual map, the underlying relationship structure resists reverting to a pure threat footing.

\textbf{Community Theme Evolution: }
Community structures mirrored edge-level shifts (Table~\ref{tab:community_frame_orientation}). In P2, diplomacy-oriented communities surged ($+17.9$ pp) while threat communities declined ($-19.3$ pp).
% Community Frame Distribution Table (LLM-classified)
\begin{table}[t]
\centering
\caption{Community Frame Orientation by Period (LLM-classified)}
\begin{tabular}{lccc}
\toprule
\textbf{Dominant Frame} & \textbf{P1} & \textbf{P2} & \textbf{P3} \\
\midrule
THREAT & 52.0\% & 32.7\% & 26.2\% \\
DIPLOMACY & 24.6\% & 42.5\% & 34.6\% \\
NEUTRAL & 13.0\% & 17.0\% & 16.8\% \\
ECONOMIC & 3.1\% & 3.3\% & 8.4\% \\
HUMANITARIAN & 7.3\% & 4.6\% & 14.0\% \\
\midrule
\textit{Total communities} & 354 & 153 & 107 \\
\bottomrule
\end{tabular}
\label{tab:community_frame_orientation}
\end{table}

In P3, rather than reverting to threat dominance, the discourse structure diversified: threat communities continued declining (26.2\%), while humanitarian ($+9.4$ pp) and economic ($+5.1$ pp) communities expanded, suggesting a broadening of the discursive agenda.

\textbf{Summary: }These analyses show that the Singapore Summit not only shifted edge-level framing (Table~\ref{tab:relationship_framing}) but also reorganized community-level thematic structure. The decline in threat-oriented framing and rise of diplomacy-oriented framing from P1 to P2, coupled with the subsequent diversification in P3, suggests that diplomatic events create cascading effects across multiple levels of discourse organization.

\subsection{Audience Response Analysis (RQ3)}
\label{subsec:audience_response}

We next examine whether the post-level effects identified in RQ1 and RQ2
are reflected in highly visible audience responses.
Our goal is not to treat comments as independent causal units,
but to assess whether framing shifts associated with diplomatic events
extend beyond agenda-setting posts into participatory discourse.

\textbf{Content-Level Audience Responses.}
We first analyze sentiment and framing patterns in audience comments.
Supplementary Difference-in-Differences (DiD) analyses on highly visible comments
($N=255{,}391$; Table~\ref{tab:dataset}) reveal significant sentiment improvements
during the Singapore Summit period
(with full DiD results reported in the Appendix).
Following the Hanoi failure, however, audience sentiment largely reverts
toward pre-summit levels, mirroring the transient pattern observed at the post level.
This contrast underscores a key distinction between affective responses
and narrative framing in audience discourse.
In particular, while sentiment is fleeting, structural framing persists—a pattern confirmed by calculating the magnitude of reversal following the Hanoi collapse relative to the initial summit shift (see details in the Appendix).

In contrast, framing patterns in comments closely mirror post-level dynamics.
Threat-oriented discourse collapses during the summit period
and does not statistically revert to pre-summit levels following the diplomatic failure
(full framing DiD results are available in the Appendix).
The persistence of diplomacy-oriented framing in comments,
alongside the transience of sentiment,
reinforces the asymmetric pattern observed at the post level
and suggests continuity in how diplomatic engagement reshapes narrative orientation.

\textbf{Structural-Level Audience Responses.}
We next assess whether these content-level patterns are reflected
in the structure of audience discourse networks.
Graph-based analyses of comment networks indicate that the post-Hanoi period
is characterized not by a return to purely confrontational structures,
but by a mixed configuration in which diplomatic connections remain salient
alongside re-emerging threats
(further structural metrics are provided in the Appendix).
In particular, diplomacy-oriented edges remain elevated relative to pre-summit baselines,
while humanitarian and sanctions-related themes expand.
Together, these patterns point to a reconfiguration of audience discourse
toward sustained, if contested, engagement rather than full reversion.

\textbf{Summary.}
Taken together, audience-level analyses reinforce the post-level conclusions
in two ways.
First, framing shifts associated with summit diplomacy are preserved
in highly visible audience interpretations, indicating that these effects
are not confined to agenda-setting posts.
Second, structural patterns in comment networks parallel the reorganization
observed at the post level, supporting a system-level interpretation
of discourse change.
Importantly, these results do not constitute independent causal estimates;
rather, they provide convergent evidence that post-level effects
are reflected in participatory

\subsection{Validation of LLM-Based Framing (RQ4)}
To evaluate whether LLM-based framing classification can approximate expert human judgment in geopolitical discourse analysis, we constructed a gold-standard benchmark of 500 Reddit posts using stratified sampling across countries (North Korea, China, Iran, Russia), time periods (P1--P3), and frames (THREAT, ECONOMIC, NEUTRAL, HUMANITARIAN, DIPLOMACY). Two domain experts (military officers with expertise in North Korea and international security) independently annotated all samples following an iterative codebook refinement process.

\textbf{Inter-Rater Reliability:} We first assess annotation reliability on the independently produced labels prior to consensus. Overall agreement between annotators was 81.8\%, with Cohen's $\kappa$ = 0.75, indicating substantial agreement.
Reliability was consistent across all framing categories, ranging from 79.6\% (DIPLOMACY) to 83.9\% (HUMANITARIAN), suggesting that the codebook definitions enabled uniform annotation quality across diverse framing contexts.

\textbf{LLM--Human Agreement:} Using the final consensus labels as ground truth, we compare LLM predictions against the benchmark annotations. Overall accuracy was 85.0\%, with Cohen's $\kappa$ = 0.79, indicating substantial agreement. Performance varied by framing category: the model achieved strong performance on all frames (F1 $\geq$ 0.75), with the highest performance on THREAT (F1 = 0.88) and NEUTRAL (F1 = 0.88). We report the full per-class precision/recall/F1 in Table~\ref{tab:llm_human_metrics}.

\begin{table}[t]
\centering
\small
\setlength{\tabcolsep}{4pt}
\caption{LLM vs.\ Human (Consensus) Validation Metrics}
\begin{tabular}{lcccc}
\toprule
\textbf{Frame} & \textbf{Prec.} & \textbf{Rec.} & \textbf{F1} & \textbf{Support} \\
\midrule
THREAT & 0.91 & 0.86 & 0.88 & 97 \\
ECONOMIC & 0.78 & 0.81 & 0.79 & 52 \\
NEUTRAL & 0.85 & 0.90 & 0.88 & 225 \\
HUMANITARIAN & 0.70 & 0.81 & 0.75 & 32 \\
DIPLOMACY & 0.88 & 0.77 & 0.82 & 108 \\
\midrule
\textbf{Overall} &  &  & \textbf{Acc. 0.85} &  \\
\textbf{Macro Avg.} & 0.83 & 0.83 & \textbf{0.83} &  \\
\bottomrule
\end{tabular}
\label{tab:llm_human_metrics}
\end{table}

\textbf{Error Analysis and Implications:} Qualitative inspection of disagreements suggests that most LLM errors arise in edge cases where multiple frames co-occur, and the primary emphasis is ambiguous. Specifically, the LLM over-applied the ``factual reporting = NEUTRAL'' heuristic to diplomatic event coverage (17 cases), classifying summits and negotiations as neutral when human annotators labeled them as DIPLOMACY. Similarly, the ``individual harm = HUMANITARIAN'' rule was triggered by posts merely mentioning individuals (8 cases), even when the overall framing was neutral. Importantly, these errors are unlikely to systematically favor one diplomatic phase over another because the benchmark was stratified across periods and countries, with error rates ranging from 11.6\% (Iran) to 18.0\% (North Korea). We provide detailed qualitative examples of these classification disagreements in Appendix~\ref{app:error_analysis}.

\textbf{Summary: }Taken together, these results support the validity of using LLM-based framing classification for large-scale causal analysis, while motivating targeted robustness checks (e.g., re-estimating DiD effects on high-confidence predictions or excluding the most ambiguous categories).

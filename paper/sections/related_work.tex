
\section{Related Work}

\subsection{Framing Dynamics in Diplomatic Discourse}

Framing theory posits that public understanding of political events is shaped by how issues are selectively presented and organized~\cite{entman1993, entman2004, iyengar1991}. Much of this foundational work has examined generic frames (e.g., conflict, economic consequences, morality) in traditional media such as newspapers and television news~\cite{semetko2000}. While these static frame categories remain influential, they were developed primarily for broadcast media and may not fully capture the dynamic, participatory nature of online discourse. More recent computational studies have begun to address this gap by examining the temporal spillover of online behavior~\cite{russo2023spillover}, community-level structural changes following platform interventions~\cite{trujillo2022make}, and the stability of discourse networks over time~\cite{vandenhole2025discourse}. Additionally, distinguishing sentiment from substantive stance has emerged as a key methodological concern, as emotional valence does not necessarily reflect substantive framing~\cite{bestvater2023sentiment}. Despite these advances, few studies have applied such approaches to measure how diplomatic events reshape the narrative positioning of an adversary over time.
% \mohit{You can add some papers and describe it. Find some newer works also, check the papers that cited them recently, like from 2022 onwards}
% \jun{Fixed: Added Russo et al. (2023), Trujillo \& Cresci (2022), Vandenhole et al. (2025), Bestvater \& Monroe (2023).}

Specifically, current literature lacks a systematic understanding of how ``dialogue'' and ``negotiation'' frames compete with entrenched ``threat'' narratives during periods of high-stakes diplomacy. More broadly, Discourse Network Analysis (DNA) has emerged as a method for studying how actors and concepts are interconnected in policy debates~\cite{leifeld2016policy, vandenhole2025discourse}, yet its application to real-time social media discourse remains limited. By examining longitudinal shifts in framing, we extend this literature to capture how diplomatic engagement and failure reshape the narrative positioning of an adversary nation over time in participatory online environments.

\subsection{Causal Analysis of Social Media Discourse}

Social media platforms have become central venues for public deliberation on international politics~\cite{baumgartner2020}. While recent work has examined online discourse toward North Korea~\cite{kim2021nktwitter}, much of this literature relies on descriptive or correlational analyses, limiting causal interpretation. Studies often document shifts in sentiment without isolating the effects of specific political events from broader trends. However, recent computational social science research has increasingly adopted quasi-experimental designs to move beyond descriptive correlations. For instance, studies examining platform interventions have employed methods such as Interrupted Time Series~\cite{ali2021understanding, jhaver2021evaluating} and Difference-in-Differences~\cite{russo2023spillover, trujillo2022make} to causally estimate the impact of moderation policies on user behavior and toxicity.
% \mohit{Add more papers, there are a lot of work on casual analysis. Be more thorough, You can find papers from my Parler paper. You will find paper in the Studies about Deplatforming section.}
% \jun{Fixed: Added Ali et al. (2021), Jhaver et al. (2021), Russo et al. (2023), Trujillo \& Cresci (2022) per Parler paper's deplatforming section.}

To address this gap in international relations scholarship, we adapt these quasi-experimental approaches, which have been increasingly applied in computational social science~\cite{ali2021understanding, jhaver2021evaluating, trujillo2022make}. For instance, Kumarswamy et al.~\cite{kumarswamy2025} utilized a Difference-in-Differences (DiD) framework to isolate the causal effects of platform policy changes on content toxicity, while Horta Ribeiro et al.~\cite{horta2023deplatforming} examined the effects of community banning on user migration and behavior.
% \mohit{Here also add more works, you can have some and then describe two of them. Add another work which i cited in intro, it is \cite{horta2023deplatforming}.}
% \jun{Fixed: Added Ali et al. (2021), Jhaver et al. (2021), Trujillo \& Cresci (2022) as listing, and described Kumarswamy et al. (2025) and Horta Ribeiro et al. (2023).}
We extend this causal inference approach to the domain of international relations. By employing a DiD design with multiple control countries (China, Iran, Russia), we isolate the specific causal impact of the Singapore and Hanoi summits on public discourse, distinguishing event-driven effects from general temporal trends.

\subsection{Large Language Models in Political Science}

Advances in large language models (LLMs) have enabled scalable analysis of political text~\cite{ziems2024}. For instance, Gilardi et al.~\cite{gilardi2023} demonstrated that ChatGPT outperforms crowd-workers for text-annotation tasks, while Törnberg~\cite{tornberg2023} established that LLMs can effectively classify political discourse with zero-shot learning.
% \mohit{Add more works and describe one or two of them.}
% \jun{Fixed: Added descriptions of Gilardi et al. (2023) and Törnberg (2023).}
However, while LLMs offer efficiency, their application in specialized geopolitical domains requires careful scrutiny. As noted by Sap et al.~\cite{sap2022annotators}, annotator identities and model biases can significantly skew automated detection. Consequently, relying on generic benchmarks or crowd-sourced labels may be insufficient for capturing complex diplomatic nuances.
% \mohit{This is a strong statement, are you sure?}
% \jun{Fixed: Toned down the statement and added Sap et al. (2022) to support the claim about annotator bias.}
Our work bridges this gap by constructing a gold-standard human annotation benchmark and explicitly validating LLM-based framing classification against domain-expert consensus, establishing a rigorous pipeline for computational diplomatic analysis.

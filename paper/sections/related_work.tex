\section{Related Work} % \mohit{Lets cut some things in related work section, Try to shorten it}
\textbf{Framing Dynamics in Diplomatic Discourse.} Framing theory posits that public understanding is shaped by selective presentation~\cite{entman1993, entman2004, iyengar1991}. While foundational work examined static frames in traditional media~\cite{semetko2000}, these categories may not capture the dynamic, participatory nature of online discourse. Recent computational studies address this by examining temporal spillovers~\cite{russo2023spillover}, structural changes~\cite{trujillo2022make}, and network stability~\cite{leifeld2016policy, vandenhole2025discourse}. Previous research further highlights how shocks shift online language~\cite{olteanu2018extremist, rizoiu2018debatenight}, how geopolitical conflicts sustain cross-platform narratives~\cite{kloo2024crossplatform, zhu2024partisan}, and the importance of distinguishing emotional valence from substantive stance~\cite{bestvater2023sentiment}. We extend this literature by measuring how diplomatic events reshape the narrative positioning of an adversary over time, addressing the lack of systematic understanding regarding how ``dialogue'' frames compete with entrenched ``threat'' narratives in participatory environments.
% \mohit{You can add some papers and describe it. Find some newer works also, check the papers that cited them recently, like from 2022 onwards}
% \jun{Fixed: Added Russo et al. (2023), Trujillo \& Cresci (2022), Vandenhole et al. (2025), Bestvater \& Monroe (2023).}

\textbf{Causal Analysis of Social Media Discourse.} Social media platforms are central venues for international political deliberation~\cite{baumgartner2020}. While existing work often relies on correlational analyses of sentiment~\cite{kim2021nktwitter}, computational social science increasingly adopts quasi-experimental designs to allow causal interpretation. Researchers have utilized Difference-in-Differences (DiD)~\cite{russo2023spillover, trujillo2022make} to isolate the effects of platform interventions on user behavior and toxicity~\cite{kumarswamy2025, horta2023deplatforming}. We adapt these causal inference approaches to international relations. By employing a DiD design with multiple control countries (China, Iran, Russia), we isolate the specific impact of the Singapore and Hanoi summits on public discourse, distinguishing event-driven effects from general temporal trends.
% \mohit{Add more papers, there are a lot of work on casual analysis. Be more thorough, You can find papers from my Parler paper. You will find paper in the Studies about Deplatforming section.}
% \jun{Fixed: Added Ali et al. (2021), Jhaver et al. (2021), Russo et al. (2023), Trujillo \& Cresci (2022) per Parler paper's deplatforming section.}
% \mohit{Here also add more works, you can have some and then describe two of them. Add another work which i cited in intro, it is \cite{horta2023deplatforming}.}
% \jun{Fixed: Added Ali et al. (2021), Jhaver et al. (2021), Trujillo \& Cresci (2022) as listing, and described Kumarswamy et al. (2025) and Horta Ribeiro et al. (2023).}

\textbf{Large Language Models in Political Science.} LLMs enable scalable political text analysis~\cite{ziems2024}, with recent work demonstrating that models like ChatGPT can classify discourse with zero-shot performance comparable to crowd-workers~\cite{gilardi2023, tornberg2023}. However, applying LLMs to specialized geopolitical domains requires scrutiny, as generic benchmarks may overlook complex diplomatic nuances or reflect underlying biases~\cite{sap2022annotators}. We address this challenge by embedding rigorous expert-defined codebook criteria directly into the LLM prompt and validating predictions against a gold-standard benchmark of domain-expert consensus, thereby aligning automated classification with domain-specific standards.
% \mohit{Add more works and describe one or two of them.}
% \jun{Fixed: Added descriptions of Gilardi et al. (2023) and Törnberg (2023).}
% \mohit{This is a strong statement, are you sure?}
% \jun{Fixed: Toned down the statement and added Sap et al. (2022) to support the claim about annotator bias.}

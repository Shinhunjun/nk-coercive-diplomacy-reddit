%==============================================================================
% RELATED WORK
%==============================================================================
\section{Related Work}

\subsection{Framing Dynamics in Diplomatic Discourse}

Framing theory posits that public understanding of political events is shaped by how issues are selectively presented and organized \cite{entman1993, entman2004}. While generic frames such as conflict, economic consequences, and morality shape foreign policy reporting \cite{semetko2000}, diplomatic summits introduce distinct cooperative narratives that require domain-specific analysis. Prior research has largely focused on traditional media or static frame categories, often overlooking the \textit{dynamic} process by which frames evolve and reorganize in response to discrete diplomatic interventions.

Specifically, current literature lacks a systematic understanding of how "dialogue" and "negotiation" frames compete with entrenched "threat" narratives during periods of high-stakes diplomacy. By examining longitudinal shifts in framing, we extend this literature to capture how diplomatic engagement and failure reshape the narrative positioning of an adversary nation over time in participatory online environments.

\subsection{Causal Analysis of Social Media Discourse}

Social media platforms have become central venues for public deliberation on international politics \cite{baumgartner2020}. While recent work has examined online discourse toward North Korea \cite{kim2021nktwitter}, much of this literature relies on descriptive or correlational analyses, limiting causal interpretation. Studies often document shifts in sentiment without isolating the effects of specific political events from broader trends.

To address this gap, recent research has begun applying quasi-experimental designs to social media contexts. For instance, Kumarswamy et al. \cite{kumarswamy2025} utilized a Difference-in-Differences (DiD) framework to isolate the causal effects of platform policy changes on content toxicity. We adapt this causal inference approach to the domain of international relations. By employing a DiD design with multiple control countries (China, Iran, Russia), we isolate the specific causal impact of the Singapore and Hanoi summits on public discourse, distinguishing event-driven effects from general temporal trends.

\subsection{Large Language Models in Political Science}

Advances in large language models (LLMs) have enabled scalable analysis of political text \cite{gilardi2023, ziems2024}. While LLMs offer efficiency for sentiment and stance classification, their validity in specialized geopolitical domains remains less established. Many studies deploy these models without systematic validation against expert human judgments. Our work bridges this methodological gap by constructing a gold-standard human annotation benchmark and explicitly validating LLM-based framing classification against expert consensus, establishing a rigorous pipeline for computational diplomatic analysis.

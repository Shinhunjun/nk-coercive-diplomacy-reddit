% Abstract

Do high-stakes diplomatic summits reshape public understanding of adversaries, or merely evoke transient emotional reactions? We investigate this question by analyzing U.S.--North Korea summit diplomacy (2018--2019) using a Difference-in-Differences (DiD) design on Reddit discourse. Using multiple control groups (China, Iran, Russia) to isolate causal effects from global geopolitical shocks, we integrate validated LLM-based framing classification with GraphRAG network analysis.

Our results reveal a striking ``ratchet effect'' characterized by asymmetric persistence. While sentiment proved \textbf{elastic}---improving during the Singapore Summit but fully reverting after the Hanoi failure---framing exhibited \textbf{plasticity}: the shift from ``threat'' to ``diplomacy'' narratives persisted partially (reversal ratio: 0.43) even after negotiations collapsed. Structurally, diplomatic engagement consolidated discourse networks (density +146\%) and established cooperative narrative connections that resisted complete reversion. These findings suggest that diplomacy, even when ultimately unsuccessful, fundamentally alters the \textit{interpretive schemas} used to discuss adversaries, leaving a lasting structural imprint that transient emotional reactions do not capture. Methodologically, we demonstrate how combining causal inference with generative AI quantification can advance the study of event-driven opinion dynamics.


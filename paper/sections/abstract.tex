% Abstract

% \mohit{I would not start abstract like this, have a very short part like intro about public opinions and then say how prior works have done and what this work is doing}
% \jun{Fixed: Added background on public opinion, prior work limitations on sentiment/survey-based measures, then what this study investigates. Also replaced density metric with edge/community proportion changes.}
    
Public opinion toward foreign adversaries shapes and constrains diplomatic options. Prior research has largely relied on sentiment analysis and survey-based measures, providing limited insight into how sustained narrative changes (beyond transient emotional reactions) might follow diplomatic engagement. This study investigates whether high-stakes diplomatic summits produce lasting shifts in how adversaries are framed in online discourse.
We analyze U.S.--North Korea summit diplomacy (2018--2019) using a Difference-in-Differences (DiD) design on Reddit discussions. Using multiple control groups (China, Iran, Russia) to isolate causal effects from concurrent geopolitical shocks, we integrate a validated \textit{Codebook LLM} framework for framing classification with graph-based discourse network analysis that examines both edge-level relationships and community-level narrative structures. Our results reveal \textit{asymmetric persistence} in framing responses to diplomacy. While sentiment proved elastic (improving during the Singapore Summit but fully reverting after the Hanoi failure), framing exhibited greater persistence: the shift from threat-oriented to diplomacy-oriented narratives was only partially reversed (reversal ratio = 0.43, 95\% CI [0.11, 0.73]). Structurally, the proportion of threat-oriented edges decreased (48\% $\rightarrow$ 28\%) while diplomacy-oriented communities nearly doubled (25\% $\rightarrow$ 43\%), and these structural shifts resisted complete reversion after diplomatic failure. These findings suggest that diplomatic success can leave a lasting imprint on how adversaries are framed in online discourse, even when subsequent negotiations fail.


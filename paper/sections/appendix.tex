%==============================================================================
% APPENDIX
%==============================================================================
% \newpage
\appendix
\section*{Appendix}
\section{LLM Classification Prompt}
\label{app:llm_prompt}

The classification was performed using GPT-4o-mini with the following system and user prompts.

\subsection*{System Prompt}
\begin{quote}
You are a political science researcher analyzing media framing of international relations. Apply the Critical Classification Rules FIRST before classifying.
\end{quote}

\subsection*{User Prompt}
You are an international relations researcher. Classify the following Reddit post into ONE of 5 framing categories.

\subsubsection*{Critical Classification Rules (Apply First!)}
These rules resolve ambiguities and edge cases.

\begin{enumerate}
    \item \textbf{No Action = NEUTRAL}: If the post is a question, hypothesis, speculation, or factual report without explicit government action, classify as NEUTRAL.
    \item \textbf{Verbal vs. Physical Actions}: If a state is \textbf{only verbally criticizing or warning} another state (not taking physical/military action), classify as DIPLOMACY, not THREAT.
    \item \textbf{Individual Harm = HUMANITARIAN}: If the harm is to \textbf{specific individuals} (protesters, defectors, refugees, civilians), classify as HUMANITARIAN, not THREAT.
    \item \textbf{Conflicting Frames = NEUTRAL}: When DIPLOMACY and THREAT (or other frames) are equally present and competing, classify as NEUTRAL.
    \item \textbf{Domestic Politics = NEUTRAL}: Commentary on domestic political issues, even if mentioning foreign countries, is NEUTRAL.
\end{enumerate}

\subsubsection*{Classification Criteria}

\paragraph{1. THREAT (Military Tension/Conflict)}
Physical military actions that increase conflict possibility.
\begin{itemize}
    \item \textbf{Include}: Military actions (missile launches, nuclear tests, military exercises, shows of force), arms buildup, military threats with NO dialogue possibility (ultimatums), cyberattacks on military/government infrastructure.
    \item \textbf{Exclude (classify as DIPLOMACY instead)}: Verbal warnings with possibility of dialogue remaining, one state verbally criticizing another's actions, requests to stop military activities.
\end{itemize}

\paragraph{2. DIPLOMACY (Diplomatic Interaction)}
Relationship adjustment through dialogue, negotiation, or verbal pressure.
\begin{itemize}
    \item \textbf{Include}: Summit meetings, diplomatic negotiations, bilateral/multilateral talks, agreements, attempts to improve/normalize relations, sanctions relief, verbal criticism/warnings between states, diplomatic pressure.
    \item \textbf{Note}: Even if a summit fails, focus on the summit itself $\rightarrow$ DIPLOMACY.
\end{itemize}

\paragraph{3. ECONOMIC (Economic Measures)}
Pressure or cooperation through economic means.
\begin{itemize}
    \item \textbf{Include}: Imposition/strengthening of economic sanctions, sanctions evasion activities, trade measures (tariffs), economic cooperation/investment.
    \item \textbf{Exclude}: Arms deals $\rightarrow$ THREAT, if main focus is diplomatic action $\rightarrow$ DIPLOMACY.
\end{itemize}

\paragraph{4. HUMANITARIAN (Humanitarian Issues)}
Human rights and individual/civilian harm.
\begin{itemize}
    \item \textbf{Include}: Human rights violations, refugee issues, humanitarian assistance/aid, war crimes, harm to individuals (protesters, defectors, refugees, civilians), cyberattacks targeting civilians.
\end{itemize}

\paragraph{5. NEUTRAL (Neutral Information)}
Cases not fitting specific frames.
\begin{itemize}
    \item \textbf{Include}: Simple factual reporting, analysis, domestic politics, questions/hypotheticals, description without explicit government action, when multiple frames are equally present.
\end{itemize}

\section{Parallel Trends Validation}
\label{app:parallel_trends}

Tables~\ref{tab:app_pt_sentiment} and~\ref{tab:app_pt_framing} present the full parallel trends test results for sentiment and framing outcomes, respectively. For sentiment, all three control groups satisfied parallel trends across both comparisons ($p > 0.05$). For framing, China and Iran satisfied parallel trends in all comparisons, while Russia exhibited a significant violation for P1$\rightarrow$P2 ($p = 0.01$), likely due to the Mueller investigation's confounding effect on threat-oriented discourse during this period. Consequently, Russia is excluded from framing analyses for the Singapore Summit effect.

\begin{table}[h]
\centering
\caption{Parallel Trends Test Results (Post Sentiment)}
\small
\begin{tabular}{llcc}
\toprule
\textbf{Comparison} & \textbf{Control} & \textbf{P-value} & \textbf{Trends} \\
\midrule
P1$\rightarrow$P2 & China & 0.99 & Valid \\
P1$\rightarrow$P2 & Iran & 0.73 & Valid \\
P1$\rightarrow$P2 & Russia & 0.14 & Valid \\
\addlinespace
P2$\rightarrow$P3 & China & 0.71 & Valid \\
P2$\rightarrow$P3 & Iran & 0.64 & Valid \\
P2$\rightarrow$P3 & Russia & 0.81 & Valid \\
\bottomrule
\end{tabular}
\label{tab:app_pt_sentiment}
\end{table}

\begin{table}[h]
\centering
\caption{Parallel Trends Test Results (Post Framing)}
\small
\begin{tabular}{llcc}
\toprule
\textbf{Comparison} & \textbf{Control} & \textbf{P-value} & \textbf{Trends} \\
\midrule
P1$\rightarrow$P2 & China & 0.06 & Valid \\
P1$\rightarrow$P2 & Iran & 0.69 & Valid \\
P1$\rightarrow$P2 & Russia & 0.01 & Invalid \\
\addlinespace
P2$\rightarrow$P3 & China & 0.40 & Valid \\
P2$\rightarrow$P3 & Iran & 0.76 & Valid \\
P2$\rightarrow$P3 & Russia & 0.30 & Valid \\
\bottomrule
\end{tabular}
\label{tab:app_pt_framing}
\end{table}


\section{Robustness Check: Alternative Framing Scale}
\label{app:robustness}

To verify that our main findings are not artifacts of the framing scale construction, we conducted a robustness check using an alternative specification that treats ECONOMIC framing as partially coercive (ECONOMIC = $-1$) rather than neutral (ECONOMIC = $0$). This alternative reflects the view that economic sanctions represent a form of coercive diplomacy.
\begin{table}[h!]
\centering
\caption{DiD Estimates: Original vs. Alternative Framing Scale}
\resizebox{\linewidth}{!}{
\begin{tabular}{llcc}
\toprule
\textbf{Event} & \textbf{Control} & \textbf{Original} & \textbf{Alternative} \\
 & & (ECON=0) & (ECON=$-1$) \\
\midrule
Singapore & China & +0.34** & +0.37** \\
(P1$\rightarrow$P2) & Iran & +0.39 & +0.77** \\
\midrule
Hanoi & China & -0.34 & -0.23 \\
(P2$\rightarrow$P3) & Iran & -0.23 & -0.22 \\
 & Russia & -0.35 & -0.36 \\
\bottomrule
\multicolumn{4}{l}{\footnotesize * $p<0.05$, ** $p<0.01$, *** $p<0.001$} \\
\end{tabular}
}
\label{tab:app_robustness}
\end{table}

The alternative specification produces similar directional effects across all comparisons, with several estimates becoming stronger under the alternative scaling. Notably, the Singapore/Iran effect becomes statistically significant under the alternative specification (+0.39 $\rightarrow$ +0.77**). These results suggest that our main specification (ECONOMIC = 0) provides a \textit{conservative} estimate of the diplomatic framing shift, and that our substantive conclusions are robust to alternative operationalizations of the framing scale.



\section{Cross-LLM Validation}
\label{app:cross_llm}

\begin{table}[h!]
\centering
\caption{Cross-LLM Validation: Agreement with GPT-4o-mini ($N=500$)}
\resizebox{\linewidth}{!}{
\begin{tabular}{llcc}
\toprule
\textbf{Model Family} & \textbf{Model Version} & \textbf{Agreement (\%)} & \textbf{Cohen's $\kappa$} \\
\midrule
Meta Llama & Llama 3.3 70B & 81.5\% & 0.71 \\
Meta Llama & Llama 3.1 8B & 75.3\% & 0.62 \\
\bottomrule
\multicolumn{4}{l}{\footnotesize Primary: GPT-4o-mini (85.0\% accuracy vs. human labels, $\kappa$=0.79)} \\
\end{tabular}
}
\label{tab:cross_llm}
\end{table}

To ensure that our framing classifications are robust across different LLM families and model sizes, we validated our primary classifier (GPT-4o-mini) against multiple alternative models. GPT-4o-mini was selected as the primary classifier based on its strong agreement with human gold labels (85.0\% accuracy, Cohen's $\kappa$ = 0.79, $N=400$). We then tested cross-model consistency by comparing other LLMs against GPT-4o-mini predictions on a stratified sample of 500 posts.

Table~\ref{tab:cross_llm} presents the agreement metrics. Among the Meta Llama models, Llama 3.3 70B achieved 81.5\% agreement ($\kappa$ = 0.71), indicating ``substantial agreement''. This high cross-family consistency, particularly between models from different developers and architectures (OpenAI vs. Meta), provides strong evidence that our classification results reflect objective discourse patterns rather than model-specific artifacts.

\subsection{Qualitative Error Analysis}
Table~\ref{tab:app_error_examples} provides qualitative examples of classification disagreements between human annotators and the LLM. The analysis highlights two primary sources of error: (1) Over-application of the ``factual reporting = NEUTRAL'' heuristic to diplomatic events, and (2) Over-sensitivity of the ``individual harm = HUMANITARIAN'' rule to keywords like ``protesters'' or ``defectors'' even when the primary framing is not humanitarian.

\begin{table*}[h]
\centering
\small
\caption{Examples of Human-LLM Classification Disagreements}
\begin{tabularx}{\textwidth}{>{\raggedright\arraybackslash}p{4.8cm} c c >{\raggedright\arraybackslash}X}
\toprule
\textbf{Post Title} & \textbf{Human} & \textbf{LLM} & \textbf{LLM Reasoning (Summary)} \\
\midrule
\multicolumn{4}{l}{\textit{Type 1: Factual Reporting vs. Diplomatic Action}} \\
\addlinespace[0.3em]
Trump calls off planned Singapore summit & DIP & NEU & Classified as a ``factual report'' about an event cancellation without explicit government action. \\
\addlinespace[0.5em]
Kim Jong Un received an `excellent' letter from Trump & DIP & NEU & Viewed as a report about a letter rather than a significant diplomatic interaction. \\
\addlinespace[0.5em]
Tillerson in Beijing set to talk on North Korea & DIP & NEU & Interpreted as a report on travel plans rather than a substantive diplomatic engagement. \\
\addlinespace[0.5em]
North Korea cuts ICBMs out of military parade & DIP & NEU & Failed to recognize the parade modification as a deliberate diplomatic signal of de-escalation. \\
\addlinespace[0.5em]
South Korea's Leader Credits Trump for North Korea Talks & DIP & NEU & Characterized as a neutral report on a leader's statement rather than an act of diplomatic recognition. \\
\midrule
\multicolumn{4}{l}{\textit{Type 2: Keyword Sensitivity (Individual Harm)}} \\
\addlinespace[0.3em]
Anonymous Hacks China... Students Trapped & THR & HUM & Focus on ``Students trapped'' outweighed the military/cyber threat context. \\
\addlinespace[0.5em]
Russia delivered 12.1 metric tons aid to Syria & DIP & HUM & Interpreted ``aid delivery'' as a humanitarian act rather than a state-led diplomatic move. \\
\addlinespace[0.5em]
Facebook Blocks \$200K Donation To Iran Quake Victims Because Sanctions & NEU & HUM & Focus on ``Quake Victims'' overrode the primary context of corporate compliance with sanctions. \\
\addlinespace[0.5em]
North Korea defectors launch 'birthday' balloons across DMZ & NEU & HUM & The keyword ``defectors'' triggered a Humanitarian classification despite the political nature of the act. \\
\addlinespace[0.5em]
Otto Warmbier: Parents of man tortured in North Korea condemn Trump & NEU & HUM & The mention of ``tortured'' led to a Humanitarian label, missing the domestic political reaction frame. \\
\bottomrule
\end{tabularx}
\label{tab:app_error_examples}
\end{table*}



\section{Extended Audience Response Analysis}
\label{app:audience_response}

This appendix provides detailed statistics and results from the supplementary analysis of audience responses (comments), which serves to validate the propagation of framing effects beyond agenda-setting posts.

\subsection{Data Summary}
We collected the full discussion trees for the top-5 most upvoted root comments for every post in the North Korea treatment group, as well as for all three control groups. This approach ensures we capture the most visible and socially reinforced reactions in the thread. Table~\ref{tab:app_comment_counts} summarizes the comment volume.

\begin{table}[h]
\centering
\caption{Audience Response Dataset Overview}
\begin{tabular}{lr}
\toprule
\textbf{Group} & \textbf{Comments} \\
\midrule
North Korea (Treatment) & 70,879 \\
China (Control 1) & 62,057 \\
Russia (Control 2) & 81,883 \\
Iran (Control 3) & 40,572 \\
\midrule
\textbf{Total} & \textbf{255,391} \\
\bottomrule
\end{tabular}
\label{tab:app_comment_counts}
\end{table}

\subsection{Parallel Trends Verification}
To validate the DiD design for audience responses, we verified the parallel trends assumption for both sentiment and framing (Tables~\ref{tab:app_comment_pt_sentiment} and~\ref{tab:app_comment_pt_framing}). Unlike the post-level analysis where Russia violated parallel trends for framing, the audience response data satisfied the parallel trends assumption across \textit{all} control groups (Russia $p=0.615$), allowing us to include all three countries in the analysis.

\begin{table}[h]
\centering
\caption{Comment Parallel Trends Results (Sentiment)}
\begin{tabular}{llcl}
\toprule
\textbf{Comparison} & \textbf{Control} & \textbf{P-value} & \textbf{Verdict} \\
\midrule
P1$\rightarrow$P2 & China & 0.51 & Valid \\
P1$\rightarrow$P2 & Iran & 0.20 & Valid \\
P1$\rightarrow$P2 & Russia & 0.61 & Valid \\
\addlinespace
P2$\rightarrow$P3 & China & 0.33 & Valid \\
P2$\rightarrow$P3 & Iran & 0.39 & Valid \\
P2$\rightarrow$P3 & Russia & 0.40 & Valid \\
\bottomrule
\end{tabular}
\label{tab:app_comment_pt_sentiment}
\end{table}

\begin{table}[h]
\centering
\caption{Comment Parallel Trends Results (Framing)}
\begin{tabular}{llcl}
\toprule
\textbf{Comparison} & \textbf{Control} & \textbf{P-value} & \textbf{Verdict} \\
\midrule
P1$\rightarrow$P2 & China & 0.57 & Valid \\
P1$\rightarrow$P2 & Iran & 0.34 & Valid \\
P1$\rightarrow$P2 & Russia & 0.62 & Valid \\
\addlinespace
P2$\rightarrow$P3 & China & 0.32 & Valid \\
P2$\rightarrow$P3 & Iran & 0.92 & Valid \\
P2$\rightarrow$P3 & Russia & 0.50 & Valid \\
\bottomrule
\end{tabular}
\label{tab:app_comment_pt_framing}
\end{table}

\subsection{Sentiment DiD Analysis Results}
Table~\ref{tab:app_comment_sentiment_did} presents the Difference-in-Differences results for comment sentiment. Consistent with post-level findings, underlying sentiment improved significantly during the summit period (P1$\rightarrow$P2). Following the Hanoi failure, sentiment showed ambiguous results: while it declined against Russia, it was sustained against China and Iran, showing partial ratchet effects even in sentiment.

\begin{table}[h]
\centering
\caption{Comment Sentiment DiD Results}
\begin{tabular}{llcc}
\toprule
\textbf{Event} & \textbf{Control} & \textbf{DiD Est.} & \textbf{P-Value} \\
\midrule
Singapore & China & +0.044 & 0.015 \\
(P1$\rightarrow$P2) & Iran & +0.042 & 0.065 \\
 & Russia & +0.053 & 0.002 \\
\midrule
Hanoi & China & -0.049 & 0.447 \\
(P2$\rightarrow$P3) & Iran & -0.000 & 0.996 \\
 & Russia & -0.029 & 0.023 \\
\bottomrule
\end{tabular}
\label{tab:app_comment_sentiment_did}
\end{table}

\subsection{Framing DiD Analysis Results}
Table~\ref{tab:app_comment_framing_did} presents the Difference-in-Differences results for audience reframing. The results confirm the ``Ratchet Effect'' hypothesis: 
\begin{itemize}
    \item \textbf{Singapore Summit (P1$\rightarrow$P2)}: Engagement produced a statistically significant shift toward diplomatic framing across valid control groups (China: $+0.480$, $p=0.007$; Iran: $+0.436$, $p=0.025$).
    \item \textbf{Hanoi Reversal (P2$\rightarrow$P3)}: Crucially, none of the control groups showed a statistically significant reversal to threat framing (all $p > 0.10$). The framing shift was sustained despite the diplomatic failure.
\end{itemize}

\begin{table}[h]
\centering
\caption{Comment Framing DiD Results: Evidence of the Ratchet Effect}
\begin{tabular}{llc}
\toprule
\textbf{Event} & \textbf{Control} & \textbf{DiD Est.} \\
\midrule
Singapore & China & +0.480** \\
(P1$\rightarrow$P2) & Iran & +0.436* \\
 & Russia & +0.318 \\
\midrule
Hanoi & China & -0.150 \\
(P2$\rightarrow$P3) & Iran & +0.067 \\
 & Russia & -0.184 \\
\bottomrule
\multicolumn{3}{l}{\footnotesize * $p<0.05$, ** $p<0.01$} \\
\end{tabular}
\label{tab:app_comment_framing_did}
\end{table}

\subsection{Structuring of Discourse: Communities and Edges}
Table~\ref{tab:app_comment_structure} details the distribution of framing within comment-derived knowledge graphs. The analysis reveals a persistent structural shift:
\begin{itemize}
    \item \textbf{Community Level}: DIPLOMACY-oriented communities doubled in P2 (28.6\%) and remained elevated in P3 (17.5\%). Notably, HUMANITARIAN communities surged in P3 (10.8\%), indicating a shift to human rights issues rather than returning to pure security threats.
    \item \textbf{Edge Level}: Threat-oriented relationships collapsed in P2 (-16.9 pp) and only partially recovered in P3, while DIPLOMACY edges remained historically high (19.7\%), confirming the structural persistence of engagement narratives.
\end{itemize}

\begin{table}[h]
\centering
\caption{Structural Framing Distribution in Comment Networks}
\begin{tabular}{lccc}
\toprule
\textbf{Frame} & \textbf{P1} & \textbf{P2} & \textbf{P3} \\
\midrule
\multicolumn{4}{l}{\textit{Community-Level}} \\
THREAT & 18.6\% & 11.3\% & 12.1\% \\
DIPLOMACY & 13.8\% & \textbf{28.6\%} & 17.5\% \\
HUMANITARIAN & 5.4\% & 4.8\% & \textbf{10.8\%} \\
NEUTRAL & 60.2\% & 51.9\% & 58.0\% \\
\midrule
\multicolumn{4}{l}{\textit{Edge-Level}} \\
THREAT & 29.1\% & 12.2\% & 18.6\% \\
DIPLOMACY & 15.9\% & \textbf{22.0\%} & 19.7\% \\
HUMANITARIAN & 3.8\% & 1.9\% & \textbf{6.5\%} \\
NEUTRAL & 45.1\% & 58.3\% & 48.5\% \\
\bottomrule
\end{tabular}
\label{tab:app_comment_structure}
\end{table}

\clearpage
\section{Temporal Trend Visualizations}
\label{app:trend_figures}

Figures~\ref{fig:app_sentiment_trends} and~\ref{fig:app_framing_trends} present the monthly time-series trends of sentiment and framing scores across all countries in our study. These visualizations provide additional context for the Difference-in-Differences analyses presented in the main text, allowing visual inspection of pre-treatment parallel trends and post-treatment divergence.

\begin{figure*}[h!]
\centering
\includegraphics[width=0.8\textwidth]{figures/fig6_sentiment_trends.pdf}
\caption{Monthly Sentiment Score Trends: NK vs. Control Groups. Vertical dashed lines mark key diplomatic events: the green line indicates the Singapore Summit (June 2018), and the red line indicates the Hanoi Summit (February 2019).}
\label{fig:app_sentiment_trends}
\end{figure*}

\begin{figure*}[h!]
\centering
\includegraphics[width=0.8\textwidth]{figures/fig2_framing_trends.pdf}
\caption{Monthly Framing Score Trends: NK vs. Control Groups. Vertical dashed lines mark key diplomatic events: the green line indicates the Singapore Summit (June 2018), and the red line indicates the Hanoi Summit (February 2019).}
\label{fig:app_framing_trends}
\end{figure*}

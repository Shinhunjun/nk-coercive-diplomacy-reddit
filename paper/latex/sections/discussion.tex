%==============================================================================
% DISCUSSION
%==============================================================================
\section{Discussion}

\subsection{Asymmetric Persistence of Diplomatic Effects}

Our findings reveal an asymmetric pattern in how diplomatic events shape online discourse toward North Korea. While the Singapore Summit produced substantial positive shifts in both sentiment (+0.10 to +0.21) and framing (+0.85 to +1.28), the subsequent failure of the Hanoi Summit resulted in comparatively smaller negative reversals (-0.06 to -0.12 in sentiment; -0.30 to -0.88 in framing).

Importantly, the negative effects of diplomatic failure did not fully offset the earlier gains from successful engagement. This pattern suggests an \emph{incomplete reversal} in public discourse, whereby positive reframing generated by diplomatic success exhibits persistence even after subsequent setbacks. Rather than returning to pre-summit baselines, discourse appears to stabilize at an intermediate state, indicating a form of \textit{path dependence} in diplomatic framing \cite{pierson2000}.

\subsection{Framing as a More Responsive Dimension than Sentiment}

A central contribution of this study lies in distinguishing between sentiment and framing as distinct dimensions of public discourse. Across all specifications, framing exhibited substantially larger effect sizes than sentiment, capturing approximately 25--32\% of its scale range compared to 5--10\% for sentiment.

This divergence suggests that diplomatic events primarily reshape \emph{how} foreign adversaries are interpreted and discussed, rather than merely influencing emotional valence. Framing reflects higher-level interpretive schemas that organize discourse, which may adjust rapidly in response to elite diplomatic signaling even when affective attitudes change more gradually. These findings highlight the importance of moving beyond sentiment-only analyses when studying public reactions to foreign policy events.

\subsection{Robustness Across Control Groups}

Our multi-control Difference-in-Differences design underscores the importance of careful control group selection in causal discourse analysis. Iran and China satisfy the parallel trends assumption across both sentiment and framing, while Russia exhibits partial violations, particularly for framing in the P1$\rightarrow$P2 comparison.

The consistency of estimated effects across control groups that meet parallel trends strengthens confidence in the causal interpretation of our results. At the same time, observed deviations for Russia illustrate how concurrent geopolitical developments can complicate inference, reinforcing the value of multiple comparison groups.

\subsection{From Content Shifts to Discourse Structure}

Beyond aggregate content changes, our findings motivate a structural interpretation of discourse dynamics. Changes in framing imply not only shifts in category prevalence but also potential reorganization of how narratives, actors, and themes are interconnected within discourse networks. This perspective motivates our GraphRAG analysis, which examines whether diplomatic events are associated with structural reconfiguration of discourse beyond surface-level content changes.

By linking framing shifts to changes in discourse structure, our approach moves toward a more holistic understanding of how diplomatic engagement reshapes public narratives in online spaces. This structural reorganization suggests that the connectivity and cohesion of narratives themselves are altered by high-profile diplomatic events \cite{vliegenthart2011}.

\subsection{Implications for Policy Communication}

These findings carry implications for diplomatic communication and public engagement strategies. First, high-profile diplomatic initiatives can rapidly alter interpretive frames within public discourse, even in adversarial contexts. Second, the persistence of positive framing following successful engagement suggests that diplomatic signaling may yield longer-lasting narrative effects than immediate sentiment responses. Finally, the divergence between framing and sentiment underscores the need for policymakers to consider narrative positioning alongside affective reactions when evaluating public responses to diplomacy.

\subsection{Limitations and Future Work}

Several limitations warrant consideration. First, Reddit users are not representative of the broader public, limiting external validity. Second, although our LLM-based framing classification is validated against expert annotations, systematic model biases may remain. Third, unobserved concurrent events may partially confound estimated effects despite our multi-control design. Future work could extend this approach to additional platforms, incorporate survey-based validation, and further explore structural discourse dynamics using network-based methods.

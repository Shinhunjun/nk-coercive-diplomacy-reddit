% Discussion
\section{Discussion}

\subsection{Asymmetry of Diplomatic Effects}

Our findings reveal an asymmetric pattern in how diplomatic events affect public opinion. The Singapore Summit produced positive shifts in both sentiment (+0.10 to +0.21) and framing (+0.85 to +1.28), while the Hanoi failure caused smaller reversals in sentiment (-0.06 to -0.12) and framing (-0.30 to -0.88).

This asymmetry suggests that successful diplomatic engagement creates durable changes in public discourse that are not fully reversed by subsequent failures---a phenomenon we term the ``ratchet effect'' in public opinion.

\subsection{Framing vs Sentiment}

An important finding is the differential magnitude of effects across our two measures:

\begin{itemize}
    \item \textbf{Framing}: Captures approximately 25-32\% of scale change (DID $\approx$ 1.0 on a -2 to +2 scale)
    \item \textbf{Sentiment}: Captures approximately 5-10\% of scale change (DID $\approx$ 0.1 on a -1 to +1 scale)
\end{itemize}

This suggests that framing---how topics are discussed---is more responsive to diplomatic events than emotional valence.

\subsection{Control Group Selection}

Our multi-control design reveals the importance of control group selection:

\begin{itemize}
    \item \textbf{Iran}: Satisfies parallel trends for both sentiment and framing across all comparisons
    \item \textbf{China}: Satisfies parallel trends with clean period definitions
    \item \textbf{Russia}: Shows some violations, particularly for framing P1$\rightarrow$P2
\end{itemize}

The consistency of results across controls that satisfy parallel trends strengthens causal interpretation.

\subsection{Implications for Policy Communication}

Our findings have implications for understanding how diplomatic initiatives are received by the public:

\begin{enumerate}
    \item Summit meetings can rapidly shift public discourse framing
    \item Failed negotiations do not fully undo positive gains
    \item Social media sentiment may lag behind framing changes
\end{enumerate}

\subsection{Limitations}

\begin{itemize}
    \item \textbf{Platform specificity}: Reddit may not represent broader public opinion
    \item \textbf{LLM classification}: While validated, potential for systematic biases
    \item \textbf{Confounder control}: Other events during study period may affect results
    \item \textbf{External validity}: Findings may be specific to U.S.-NK context
\end{itemize}

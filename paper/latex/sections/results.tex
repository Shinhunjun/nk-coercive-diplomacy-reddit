% Results
\section{Results}

\subsection{RQ1: Summit Effects on Sentiment}

\subsubsection{Parallel Trends Validation}

\begin{table}[h]
\centering
\caption{Parallel Trends Test Results (Sentiment)}
\begin{tabular}{llccc}
\hline
\textbf{Comparison} & \textbf{Control} & \textbf{P-value} & \textbf{Satisfied} \\
\hline
P1$\rightarrow$P2 & China & 0.99 & \checkmark \\
P1$\rightarrow$P2 & Iran & 0.83 & \checkmark \\
P1$\rightarrow$P2 & Russia & 0.20 & \checkmark \\
P2$\rightarrow$P3 & China & 0.69 & \checkmark \\
P2$\rightarrow$P3 & Iran & 0.75 & \checkmark \\
P2$\rightarrow$P3 & Russia & 0.88 & \checkmark \\
\hline
\end{tabular}
\label{tab:pt_sentiment}
\end{table}

\subsubsection{Singapore Summit Effect (P1$\rightarrow$P2)}

\begin{table}[h]
\centering
\caption{Sentiment DID: Singapore Summit Effect}
\begin{tabular}{lccc}
\hline
\textbf{Control} & \textbf{DID} & \textbf{P-value} & \textbf{Parallel Trends} \\
\hline
China & \textbf{+0.21} & $<$0.0001 & \checkmark \\
Iran & \textbf{+0.10} & $<$0.0001 & \checkmark \\
Russia & \textbf{+0.14} & $<$0.0001 & \checkmark \\
\hline
\end{tabular}
\label{tab:did_sentiment_p1p2}
\end{table}

\subsubsection{Hanoi Summit Effect (P2$\rightarrow$P3)}

\begin{table}[h]
\centering
\caption{Sentiment DID: Hanoi Summit Effect}
\begin{tabular}{lccc}
\hline
\textbf{Control} & \textbf{DID} & \textbf{P-value} & \textbf{Parallel Trends} \\
\hline
China & \textbf{-0.11} & $<$0.0001 & \checkmark \\
Iran & \textbf{-0.06} & 0.001 & \checkmark \\
Russia & \textbf{-0.12} & $<$0.0001 & \checkmark \\
\hline
\end{tabular}
\label{tab:did_sentiment_p2p3}
\end{table}

\subsection{RQ2: Rally-Round-the-Flag Patterns}

The data reveals an asymmetric rally effect:

\begin{enumerate}
    \item \textbf{Summit Success Effect}: Singapore Summit produced a sentiment improvement of +0.10 to +0.21
    \item \textbf{Summit Failure Effect}: Hanoi Summit produced a sentiment decline of -0.06 to -0.12
    \item \textbf{Net Effect}: The positive gains slightly outweigh the negative reversal
\end{enumerate}

This suggests that while diplomatic failures cause sentiment reversals, the gains from successful summits are not fully erased---indicating a partial ``ratchet effect'' in public opinion.

\subsection{Summit Effects on Framing}

\subsubsection{Parallel Trends Validation}

\begin{table}[h]
\centering
\caption{Parallel Trends Test Results (Framing)}
\begin{tabular}{llccc}
\hline
\textbf{Comparison} & \textbf{Control} & \textbf{P-value} & \textbf{Satisfied} \\
\hline
P1$\rightarrow$P2 & China & 0.12 & \checkmark \\
P1$\rightarrow$P2 & Iran & 0.77 & \checkmark \\
P1$\rightarrow$P2 & Russia & 0.03 & $\times$ \\
P2$\rightarrow$P3 & China & 0.41 & \checkmark \\
P2$\rightarrow$P3 & Iran & 0.79 & \checkmark \\
P2$\rightarrow$P3 & Russia & 0.51 & \checkmark \\
\hline
\end{tabular}
\label{tab:pt_framing}
\end{table}

\subsubsection{Singapore Summit Effect on Framing (P1$\rightarrow$P2)}

\begin{table}[h]
\centering
\caption{Framing DID: Singapore Summit Effect}
\begin{tabular}{lccc}
\hline
\textbf{Control} & \textbf{DID} & \textbf{P-value} & \textbf{Parallel Trends} \\
\hline
China & \textbf{+1.28} & $<$0.0001 & \checkmark \\
Iran & \textbf{+0.85} & $<$0.0001 & \checkmark \\
Russia & +1.04 & $<$0.0001 & $\times$ \\
\hline
\end{tabular}
\label{tab:did_framing_p1p2}
\end{table}

\subsubsection{Hanoi Summit Effect on Framing (P2$\rightarrow$P3)}

\begin{table}[h]
\centering
\caption{Framing DID: Hanoi Summit Effect}
\begin{tabular}{lccc}
\hline
\textbf{Control} & \textbf{DID} & \textbf{P-value} & \textbf{Parallel Trends} \\
\hline
China & \textbf{-0.88} & $<$0.0001 & \checkmark \\
Iran & \textbf{-0.30} & 0.003 & \checkmark \\
Russia & \textbf{-0.83} & $<$0.0001 & \checkmark \\
\hline
\end{tabular}
\label{tab:did_framing_p2p3}
\end{table}

\subsection{RQ3: Network Structure Changes}

[TO BE COMPLETED: GraphRAG Analysis Results]

\subsection{Visualizations}

\begin{figure}[h]
\centering
\includegraphics[width=\textwidth]{figures/fig1_timeline.png}
\caption{Research Timeline and Key Events}
\label{fig:timeline}
\end{figure}

\begin{figure}[h]
\centering
\includegraphics[width=\textwidth]{figures/fig2_framing_trends.png}
\caption{Monthly Framing Score Trends: NK vs Control Groups}
\label{fig:framing_trends}
\end{figure}

\begin{figure}[h]
\centering
\includegraphics[width=\textwidth]{figures/fig3_frame_distribution.png}
\caption{Frame Distribution by Period (North Korea)}
\label{fig:frame_distribution}
\end{figure}

\begin{figure}[h]
\centering
\includegraphics[width=\textwidth]{figures/fig4_did_visualization.png}
\caption{Difference-in-Differences Visualization}
\label{fig:did}
\end{figure}

\begin{figure}[h]
\centering
\includegraphics[width=\textwidth]{figures/fig5_frame_heatmap.png}
\caption{Frame Category Heatmap (North Korea)}
\label{fig:heatmap}
\end{figure}

\begin{figure}[h]
\centering
\includegraphics[width=\textwidth]{figures/fig6_sentiment_trends.png}
\caption{Monthly Sentiment Score Trends}
\label{fig:sentiment_trends}
\end{figure}

\begin{figure}[h]
\centering
\includegraphics[width=\textwidth]{figures/fig7_sentiment_framing_corr.png}
\caption{Sentiment vs Framing Correlation}
\label{fig:correlation}
\end{figure}

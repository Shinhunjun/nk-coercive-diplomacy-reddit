% Results
\section{Results}

% Timeline figure at the beginning
\begin{figure*}[t]
\centering
\includegraphics[width=0.9\textwidth]{figures/fig1_timeline.png}
\caption{Research Timeline and Key Events. Shaded regions indicate analysis periods: P1 (gray), Transition (yellow), P2 (green), P3 (red).}
\label{fig:timeline}
\end{figure*}

\subsection{RQ1: Content Changes (Sentiment & Framing)}

\subsubsection{Impact on Sentiment}

\begin{table}[t]
\centering
\caption{Parallel Trends Test Results (Sentiment)}
\begin{tabular}{llcc}
\hline
\textbf{Comparison} & \textbf{Control} & \textbf{P-value} & \textbf{OK} \\
\hline
P1$\rightarrow$P2 & China & 0.99 & \checkmark \\
P1$\rightarrow$P2 & Iran & 0.83 & \checkmark \\
P2$\rightarrow$P3 & China & 0.69 & \checkmark \\
P2$\rightarrow$P3 & Iran & 0.75 & \checkmark \\
\hline
\end{tabular}
\label{tab:pt_sentiment}
\end{table}

The Singapore Summit (P1$\rightarrow$P2) produced a significant positive shift in sentiment across all control groups (+0.10 to +0.21, p < 0.001). Conversely, the Hanoi Summit failure (P2$\rightarrow$P3) resulted in a negative shift (-0.06 to -0.12, p < 0.001).

% Sentiment trends figure near sentiment results
\begin{figure}[t]
\centering
\includegraphics[width=0.95\columnwidth]{figures/fig6_sentiment_trends.png}
\caption{Monthly Sentiment Score Trends}
\label{fig:sentiment_trends}
\end{figure}

\subsubsection{Impact on Framing}

We observed similar patterns in framing, where positive values indicate diplomatic framing and negative values indicate threat framing.

% Framing trends figure near framing results
\begin{figure*}[t]
\centering
\includegraphics[width=0.9\textwidth]{figures/fig2_framing_trends.png}
\caption{Monthly Framing Score Trends: NK vs Control Groups. Vertical lines mark key events.}
\label{fig:framing_trends}
\end{figure*}

\begin{table}[t]
\centering
\caption{Framing DID Results Summary}
\begin{tabular}{llcc}
\hline
\textbf{Event} & \textbf{Control} & \textbf{DID Est.} & \textbf{P-val} \\
\hline
Singapore & China & \textbf{+1.28} & $<$0.001 \\
(P1$\rightarrow$P2) & Iran & \textbf{+0.85} & $<$0.001 \\
\hline
Hanoi & China & \textbf{-0.88} & $<$0.001 \\
(P2$\rightarrow$P3) & Iran & \textbf{-0.30} & 0.003 \\
\hline
\end{tabular}
\label{tab:did_framing_summary}
\end{table}

The Singapore Summit led to a strong shift toward diplomatic frames (+0.85 to +1.28), while the Hanoi Summit caused a partial reversion toward threat frames (-0.30 to -0.88).

% DID visualization figure
\begin{figure*}[t]
\centering
\includegraphics[width=0.9\textwidth]{figures/fig4_did_visualization.png}
\caption{Difference-in-Differences Visualization. (A) Singapore Summit Effect, (B) Hanoi Summit Effect.}
\label{fig:did}
\end{figure*}

\subsection{RQ2: Structural Changes (GraphRAG)}

[TO BE COMPLETED: GraphRAG Analysis Results]

%==============================================================================
% RESULTS
%==============================================================================
\section{Results}

% Timeline figure at the beginning
\begin{figure*}[t]
\centering
\includegraphics[width=0.9\textwidth]{figures/fig1_timeline.png}
\caption{Research Timeline and Key Events. Shaded regions indicate analysis periods: P1 (gray), Transition (yellow), P2 (green), P3 (red).}
\label{fig:timeline}
\end{figure*}

\subsection{Content Changes in Online Discourse (RQ1)}

We first examine whether high-stakes diplomatic events causally reshape the content of online discourse toward North Korea. While sentiment captures short-term emotional responses, our primary focus is on framing, which reflects how North Korea is narratively positioned within geopolitical discourse. Together, these measures allow us to assess both affective and interpretive shifts following diplomatic events.

\subsubsection{Impact on Sentiment}

\begin{table}[t]
\centering
\caption{Parallel Trends Test Results (Sentiment)}
\begin{tabular}{llcc}
\hline
\textbf{Comparison} & \textbf{Control} & \textbf{P-value} & \textbf{OK} \\
\hline
P1$\rightarrow$P2 & China & 0.99 & \checkmark \\
P1$\rightarrow$P2 & Iran & 0.83 & \checkmark \\
P2$\rightarrow$P3 & China & 0.69 & \checkmark \\
P2$\rightarrow$P3 & Iran & 0.75 & \checkmark \\
\hline
\end{tabular}
\label{tab:pt_sentiment}
\end{table}

The parallel trends tests indicate no significant pre-treatment differences between North Korea and the control countries, supporting the validity of the Difference-in-Differences design.

The Singapore Summit (P1$\rightarrow$P2) produced a statistically significant positive shift in sentiment toward North Korea across all control groups, with estimated effects ranging from +0.10 to +0.21 (p < 0.001). In contrast, the failure of the Hanoi Summit (P2$\rightarrow$P3) resulted in a significant negative shift in sentiment, ranging from -0.06 to -0.12 (p < 0.001).

\begin{figure}[t]
\centering
\includegraphics[width=0.95\columnwidth]{figures/fig6_sentiment_trends.png}
\caption{Monthly Sentiment Score Trends}
\label{fig:sentiment_trends}
\end{figure}

While these sentiment shifts are statistically significant, their magnitude remains modest relative to the scale of measurement, suggesting that diplomatic events influence affective tone without fundamentally transforming how North Korea is discussed.

\subsubsection{Impact on Framing}

We next examine changes in framing, where higher values indicate diplomacy-oriented narratives and lower values indicate threat-oriented narratives.

\begin{figure*}[t]
\centering
\includegraphics[width=0.9\textwidth]{figures/fig4_did_visualization.png}
\caption{Difference-in-Differences Visualization. (A) Singapore Summit Effect, (B) Hanoi Summit Effect.}
\label{fig:did}
\end{figure*}

\begin{figure*}[t]
\centering
\includegraphics[width=0.9\textwidth]{figures/fig2_framing_trends.png}
\caption{Monthly Framing Score Trends: NK vs. Control Groups. Vertical lines mark key events.}
\label{fig:framing_trends}
\end{figure*}

\begin{table}[t]
\centering
\caption{Framing Difference-in-Differences Results}
\begin{tabular}{llcc}
\hline
\textbf{Event} & \textbf{Control} & \textbf{DID Est.} & \textbf{P-value} \\
\hline
Singapore & China & \textbf{+1.28} & $<$0.001 \\
(P1$\rightarrow$P2) & Iran & \textbf{+0.85} & $<$0.001 \\
\hline
Hanoi & China & \textbf{-0.88} & $<$0.001 \\
(P2$\rightarrow$P3) & Iran & \textbf{-0.30} & 0.003 \\
\hline
\end{tabular}
\label{tab:did_framing_summary}
\end{table}

The Singapore Summit led to a pronounced shift toward diplomacy-oriented framing, with DID estimates ranging from +0.85 to +1.28 across control groups. Following the collapse of the Hanoi Summit, framing shifted back toward threat-oriented narratives; however, this reversion was partial, with effect sizes (-0.30 to -0.88) substantially smaller than the preceding gains.

Notably, the magnitude of framing shifts substantially exceeds that of sentiment changes, indicating that diplomatic events more strongly reshape how North Korea is framed than how it is emotionally evaluated in online discourse.

Taken together, these findings demonstrate that high-stakes diplomatic summits significantly reshape the content of online discourse, particularly in terms of framing. We next examine whether these content-level shifts are accompanied by structural reorganization in how narratives are connected within discourse networks.

\subsection{Structural Reorganization of Discourse Networks (RQ2)}

To assess whether observed framing changes are reflected in the organization of discourse itself, we analyze discourse networks constructed from Reddit discussions across the three diplomatic phases. Using GraphRAG~\cite{edge2024graphrag}, we extract entities and relationships from 10,659 documents, constructing knowledge graphs that capture how actors, events, and concepts are interconnected in public discourse.

\subsubsection{Network Topology Changes}

Table~\ref{tab:network_topology} presents network-level metrics across the three periods. Despite a substantial reduction in the number of entities (nodes) and relationships (edges) from P1 to P3, network density---the proportion of possible connections that are realized---increased markedly (+146\% from P1 to P2, +16\% from P2 to P3). This densification indicates that discourse became more focused and interconnected around a smaller set of central actors and themes following the Singapore Summit.

The number of connected components decreased from 25 to 16, suggesting greater integration of previously fragmented discussion threads into a more unified discourse network. Clustering coefficients remained relatively stable, indicating that local community structure was preserved even as global connectivity increased.

\begin{table}[t]
\centering
\caption{Network Topology Metrics Across Diplomatic Periods}
\begin{tabular}{lcccccc}
\hline
\textbf{Metric} & \textbf{P1} & \textbf{P2} & \textbf{P3} & \textbf{P1$\rightarrow$P2} & \textbf{P2$\rightarrow$P3} \\
\hline
Nodes & 2,656 & 1,043 & 879 & $-60.7\%$ & $-15.7\%$ \\
Edges & 4,552 & 1,726 & 1,429 & $-62.1\%$ & $-17.2\%$ \\
Density ($\times10^{-3}$) & 1.3 & 3.2 & 3.7 & $+146\%$ & $+16\%$ \\
Avg Degree & 3.43 & 3.31 & 3.25 & $-3.4\%$ & $-1.8\%$ \\
Clustering Coef. & 0.215 & 0.198 & 0.210 & $-7.9\%$ & $+6.0\%$ \\
Components & 25 & 23 & 16 & $-8.0\%$ & $-30.4\%$ \\
\hline
\end{tabular}
\label{tab:network_topology}
\end{table}


\begin{figure*}[t]
\centering
\includegraphics[width=0.9\textwidth]{figures/fig_network_topology.png}
\caption{Network Topology Metrics Across Diplomatic Periods. (A) Network density increased despite fewer nodes, indicating discourse concentration. (B) Entity count declined as discussions focused on key actors. (C) Community count decreased, reflecting narrative consolidation.}
\label{fig:network_topology}
\end{figure*}

\subsubsection{Relationship Framing Shifts}

Beyond structural changes, we analyze the semantic content of relationships by classifying each edge based on keywords in its description. Table~\ref{tab:relationship_framing} and Figure~\ref{fig:framing_distribution} present the distribution of threat-oriented, peace-oriented, and neutral framings across periods.

\begin{table}[t]
\centering
\setlength{\tabcolsep}{4pt}
\caption{Relationship Framing Distribution}
\small
\begin{tabular}{lccccc}
\hline
\textbf{Frame} & \textbf{P1} & \textbf{P2} & \textbf{P3} & \textbf{$\Delta_1$} & \textbf{$\Delta_2$} \\
\hline
Threat & 61.5\% & 42.4\% & 40.9\% & \textbf{$-19$pp} & $-1$pp \\
Peace & 7.8\% & 24.7\% & 18.6\% & \textbf{$+17$pp} & $-6$pp \\
Neutral & 30.8\% & 32.9\% & 40.5\% & $+2$pp & $+8$pp \\
\hline
\end{tabular}
\begin{flushleft}
\footnotesize{$\Delta_1$: P1$\rightarrow$P2; $\Delta_2$: P2$\rightarrow$P3; pp = percentage points}
\end{flushleft}
\label{tab:relationship_framing}
\end{table}


The results reveal a pronounced shift away from threat framing following the Singapore Summit: threat-oriented relationships decreased from 61.5\% in P1 to 42.4\% in P2 ($-19.1$ percentage points), while peace-oriented relationships more than tripled from 7.8\% to 24.7\% ($+16.9$ pp).

Critically, the Hanoi Summit failure produced only a partial reversion: threat framing declined marginally to 40.9\% ($-1.5$ pp), and peace framing decreased to 18.6\% ($-6.1$ pp). This asymmetric pattern---where diplomatic gains are largely preserved despite event failure---provides network-level evidence for the ``ratchet effect'' observed in content-level framing analysis (RQ1).

\begin{figure}[t]
\centering
\includegraphics[width=0.95\columnwidth]{figures/fig_framing_distribution.png}
\caption{Relationship Framing Distribution Across Periods. The Singapore Summit (P1$\rightarrow$P2) produced a 19pp decline in threat framing and 17pp increase in peace framing. The Hanoi failure (P2$\rightarrow$P3) caused only partial reversion.}
\label{fig:framing_distribution}
\end{figure}

\subsubsection{Community Theme Evolution}

Using hierarchical community detection, we identified 354, 153, and 107 communities in P1, P2, and P3 respectively. Table~\ref{tab:keyword_evolution} shows the evolution of dominant keywords in community descriptions across periods.

\begin{table}[t]
\centering
\setlength{\tabcolsep}{3pt}
\caption{Top Keywords in Community Summaries}
\small
\begin{tabular}{clll}
\hline
\textbf{\#} & \textbf{P1} & \textbf{P2} & \textbf{P3} \\
\hline
1 & military (252) & nuclear (91) & intl. (58) \\
2 & nuclear (203) & military (84) & diplomatic (44) \\
3 & intl. (152) & intl. (67) & military (38) \\
4 & security (117) & diplomatic (62) & kim (37) \\
5 & diplomatic (96) & relations (50) & trump (33) \\
\hline
\multicolumn{4}{l}{\footnotesize{New in P2: \textit{efforts, ongoing, united}}} \\
\multicolumn{4}{l}{\footnotesize{New in P3: \textit{trump, political, complex}}} \\
\hline
\end{tabular}
\label{tab:keyword_evolution}
\end{table}


In P1, ``military'' was the most frequent keyword, reflecting the predominant threat-oriented discourse. By P2, ``nuclear'' and ``diplomatic'' rose in prominence, corresponding to summit-related discussions about denuclearization. Notably, new keywords emerged in P2 (``efforts,'' ``ongoing,'' ``united,'' ``states'') that were absent in P1, suggesting the introduction of cooperation-oriented vocabulary.

In P3, while some threat-related terms returned, diplomatic framing persisted: ``diplomatic'' remained among the top keywords, and new terms (``trump,'' ``political'') emerged, indicating continued focus on leadership-level negotiations rather than a return to purely military discourse.

\begin{figure*}[t]
\centering
\includegraphics[width=0.9\textwidth]{figures/fig_keyword_evolution.png}
\caption{Top Keywords in Community Summaries by Period. (A) P1 dominated by ``military'' and ``security.'' (B) P2 shows rise of ``diplomatic'' and ``efforts.'' (C) P3 maintains diplomatic vocabulary while adding leadership-focused terms.}
\label{fig:keyword_evolution}
\end{figure*}

Taken together, these structural analyses demonstrate that the Singapore Summit not only shifted the content of discourse (RQ1) but also reorganized its underlying network structure, concentrating attention on diplomatic themes and actors. The persistence of these structural changes following the Hanoi failure reinforces the ``ratchet effect'' hypothesis: high-profile diplomatic engagement creates durable shifts in how public discourse is organized, even when negotiations fail.




\subsection{Validation of LLM-Based Framing (RQ3)}

To evaluate whether LLM-based framing classification can approximate expert human judgment in geopolitical discourse analysis, we constructed a gold-standard benchmark of 1,330 Reddit posts using stratified sampling across countries (North Korea, China, Iran, Russia), time periods (P1--P3), and frames (THREAT, ECONOMIC, NEUTRAL, HUMANITARIAN, DIPLOMACY). Two domain experts (military officers with expertise in North Korea and international security) independently annotated all samples following an iterative codebook refinement process.

\subsubsection{Inter-Rater Reliability}

We first assess annotation reliability on the independently produced labels prior to consensus. Overall agreement between annotators was [TBD]\%, with Cohen's $\kappa$ = [TBD], indicating [TBD: moderate/substantial/almost perfect] agreement under conventional interpretations (Landis \& Koch, 1977). Reliability was highest for [TBD: e.g., THREAT/DIPLOMACY] and comparatively lower for [TBD: e.g., NEUTRAL/ECONOMIC], consistent with the greater conceptual overlap among non-extreme frames.

\subsubsection{LLM--Human Agreement}

Using the final consensus labels as ground truth, we compare LLM predictions against the benchmark annotations. Overall accuracy was [TBD], with macro-F1 of [TBD]. Performance varied by frame category: the model achieved strong performance on salient frames (e.g., THREAT and DIPLOMACY) but showed more confusion among adjacent categories such as NEUTRAL vs.\ ECONOMIC and ECONOMIC vs.\ HUMANITARIAN. We report the full confusion matrix and per-class precision/recall/F1 in Table~\ref{tab:llm_human_metrics}.

\begin{table*}[t]
\centering
\caption{LLM vs.\ Human (Consensus) Validation Metrics}
\begin{tabular}{lcccc}
\hline
\textbf{Frame} & \textbf{Prec.} & \textbf{Rec.} & \textbf{F1} & \textbf{Support} \\
\hline
THREAT & [TBD] & [TBD] & [TBD] & [TBD] \\
ECONOMIC & [TBD] & [TBD] & [TBD] & [TBD] \\
NEUTRAL & [TBD] & [TBD] & [TBD] & [TBD] \\
HUMANITARIAN & [TBD] & [TBD] & [TBD] & [TBD] \\
DIPLOMACY & [TBD] & [TBD] & [TBD] & [TBD] \\
\hline
\textbf{Overall} &  &  & \textbf{Acc. [TBD]} &  \\
\textbf{Macro Avg.} & [TBD] & [TBD] & \textbf{[TBD]} &  \\
\hline
\end{tabular}
\label{tab:llm_human_metrics}
\end{table*}


\subsubsection{Error Analysis and Implications}

Qualitative inspection of disagreements suggests that most LLM errors arise in edge cases where multiple frames co-occur and the primary emphasis is ambiguous (e.g., sanction policy described alongside civilian harm, or diplomatic negotiations described in the context of military escalation). Importantly, these errors are unlikely to systematically favor one diplomatic phase over another because the benchmark was stratified across periods and countries. Taken together, these results support the validity of using LLM-based framing classification for large-scale causal analysis, while motivating targeted robustness checks (e.g., re-estimating DID effects on high-confidence predictions or excluding the most ambiguous categories).


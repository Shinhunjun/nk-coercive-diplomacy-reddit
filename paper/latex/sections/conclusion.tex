% Conclusion
\section{Conclusion}

This study provides causal evidence of how high-stakes diplomatic summits affect public opinion framing on social media. Using a Difference-in-Differences design with validated LLM-based classification, we find that:

\begin{enumerate}
    \item The 2018 Singapore Summit produced significant positive shifts in both sentiment and framing toward North Korea.
    \item The 2019 Hanoi Summit failure triggered partial reversals, but did not fully undo the positive gains, confirming a ``ratchet effect'' in public discourse.
    \item Framing shifts (28.5\% of scale) were an order of magnitude larger than sentiment shifts (3.5\%), indicating that diplomatic events are more effective at changing the \textit{interpretive frame} than the \textit{affective tone}.
\end{enumerate}

These findings offer a critical policy implication: diplomatic engagement can successfully expand the ``policy space'' by shifting how an adversary is framed, even when underlying public sentiment remains largely negative. For policymakers, this suggests that the success of a diplomatic initiative should be evaluated not just by affective approval, but by its ability to transition public discourse from binary threat-oriented narratives toward procedural and diplomacy-oriented frameworks.

Our methodology---combining LLM classification with expert validation and rigorous causal inference---provides a template for studying public opinion dynamics around international events. Future work should extend this analysis to other diplomatic contexts and examine the long-term persistence of narrative framing in the face of sustained diplomatic stalemates.

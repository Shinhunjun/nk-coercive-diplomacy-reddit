% Introduction
\section{Introduction}

The historic 2018 Singapore Summit between U.S. President Donald Trump and North Korean leader Kim Jong Un marked the first-ever meeting between sitting leaders of the two nations. This diplomatic breakthrough, followed by the collapsed Hanoi Summit in February 2019, provides a unique natural experiment to study how high-stakes international events shape online public opinion.

\subsection{Research Questions}

\begin{itemize}
    \item \textbf{RQ1}: How do high-stakes diplomatic summits affect public opinion framing on social media, and do failed negotiations lead to symmetric reversals of opinion gains?
    \item \textbf{RQ2}: Can social media discourse predict or reflect the ``rally-round-the-flag'' effect during diplomatic engagement cycles?
    \item \textbf{RQ3}: How do discourse network structures change across diplomatic event cycles?
\end{itemize}

\subsection{Contributions}

\begin{enumerate}
    \item \textbf{Methodological}: We introduce a validated pipeline combining LLM-based framing analysis with human expert annotation (military officers) for domain-specific content classification.
    \item \textbf{Empirical}: We provide causal evidence of diplomatic events' impact on online discourse using Difference-in-Differences with multiple control groups.
    \item \textbf{Theoretical}: We extend the ``rally-round-the-flag'' literature to social media contexts, examining asymmetry between diplomatic successes and failures.
\end{enumerate}

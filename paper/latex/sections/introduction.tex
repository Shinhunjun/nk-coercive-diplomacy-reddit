% Introduction
\section{Introduction}

The historic 2018 Singapore Summit between U.S. President Donald Trump and North Korean leader Kim Jong Un marked the first-ever meeting between sitting leaders of the two nations. This diplomatic breakthrough, followed by the collapsed Hanoi Summit in February 2019, provides a unique natural experiment to study how high-stakes international events shape online public opinion.

\subsection{Research Questions}

We investigate the causal impact of diplomatic summits on public opinion through two primary lenses:

\begin{itemize}
    \item \textbf{RQ1 (Adversary Perception)}: How do high-stakes diplomatic summits affect public opinion toward adversary nations? 
    \begin{itemize}
        \item Does the \textit{content} of discourse (framing/sentiment) change significantly?
        \item Does the \textit{structure} of discourse networks shift to support these framing changes? (Analyzed via GraphRAG)
    \end{itemize}
    
    \item \textbf{RQ2 (Dynamics \& Asymmetry)}: Do failed negotiations lead to symmetric reversals of opinion gains, or do positive effects persist? (Testing the ``ratchet effect'')
\end{itemize}

\noindent\textit{Note on Scope: While traditional ``rally-round-the-flag'' literature focuses on domestic leader approval, this study focuses on adversary perception. We are currently collecting additional data to analyze leader-centric sentiment (e.g., perception of President Trump) as a distinct follow-up study.}

\subsection{Contributions}

\begin{enumerate}
    \item \textbf{Methodological}: A validated pipeline combining LLM-based framing analysis with human expert annotation (military officers).
    \item \textbf{Empirical}: Causal evidence of summit effects on adversary perception using Difference-in-Differences with multiple controls.
    \item \textbf{Theoretical}: Evidence of asymmetric effects---diplomatic successes produce larger gains than failures produce reversals.
\end{enumerate}

% Introduction
\section{Introduction}

The historic 2018 Singapore Summit between U.S. President Donald Trump and North Korean leader Kim Jong Un marked the first-ever meeting between sitting leaders of the two nations. This diplomatic breakthrough, followed by the collapsed Hanoi Summit in February 2019, provides a unique natural experiment to study how high-stakes international events shape online public opinion.

\subsection{Research Questions}

We investigate the causal impact of diplomatic summits on public opinion through two complementary dimensions:

\begin{itemize}
    \item \textbf{RQ1 (Content Change)}: How do high-stakes diplomatic summits affect the \textit{content} of public opinion (framing and sentiment) toward adversary nations?
    \item \textbf{RQ2 (Structural Change)}: How does the underlying \textit{network structure} of discourse change across different phases of diplomatic engagement?
\end{itemize}

\subsection{Contributions}

\begin{enumerate}
    \item \textbf{Methodological}: A validated pipeline combining LLM-based framing analysis with human expert annotation (military officers).
    \item \textbf{Empirical}: Causal evidence of summit effects on adversary perception using Difference-in-Differences with multiple controls.
    \item \textbf{Theoretical}: Evidence of asymmetric effects---diplomatic successes produce larger gains than failures produce reversals.
\end{enumerate}

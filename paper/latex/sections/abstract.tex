% Abstract

High-stakes diplomatic summits can substantially reshape how foreign adversaries are discussed in public discourse, yet the causal mechanisms underlying these shifts remain poorly understood in online environments. This study examines how the 2018 Singapore Summit and the failed 2019 Hanoi Summit affected the framing of North Korea on Reddit. Leveraging over 35,000 Reddit posts from 2017--2019, we combine a Difference-in-Differences design with large language model (LLM)-based framing classification validated against expert human annotations by military officers.

Our results show that the Singapore Summit produced a significant shift away from threat-oriented discourse toward diplomacy-oriented framing, accompanied by more modest improvements in sentiment. In contrast, the Hanoi Summit failure triggered a partial reversal in both framing and sentiment, though the earlier framing gains were not fully undone. Beyond content-level change, we find evidence that diplomatic events reorganize the structure of online discourse by altering the connectivity and prominence of dominant narratives.

Together, these findings suggest that successful diplomatic engagement can induce asymmetric and durable changes in how adversaries are framed in online discourse. Methodologically, this study demonstrates how validated LLM-based framing analysis combined with causal inference can advance the study of event-driven opinion dynamics in computational social science.

